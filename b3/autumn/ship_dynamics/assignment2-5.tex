%==============================================================================
% LaTeXの処理に関するおまじない (マジックコメント)
%==============================================================================
% !TEX encoding = UTF-8
%==============================================================================
% ドキュメントクラス
%==============================================================================
\documentclass[11pt]{ltjsarticle}

%==============================================================================
%【必須級】便利なパッケージたち
%==============================================================================

%----- ページ設定 -----
\usepackage[
    top=20mm,
    bottom=20mm,
    left=20mm,
    right=20mm
]{geometry}

%----- フォント設定 -----
\usepackage[T1]{fontenc}
\usepackage{newtxtext}
\usepackage{courier} 
\usepackage{newtxmath}
\usepackage{textcomp}
\usepackage{newtxtt}

%----- 数式関連 -----
\usepackage{amsmath}
\usepackage[detect-all]{siunitx}

%----- 図表・画像関連 -----
\usepackage{graphicx}
\usepackage{here}
\usepackage{booktabs}
\usepackage{float}

%----- ソースコード表示 -----
\usepackage{listings}
\lstset{
    basicstyle=\ttfamily\small,
    breaklines=true,
    frame=single,
    commentstyle={\itshape \color[gray]{0.5}},
    keywordstyle={\bfseries \color{blue}},
    stringstyle={\color{red}},
    showstringspaces=false,
    numbers=left,
    numberstyle=\tiny\color[gray]{0.5},
    captionpos=b
}

%----- その他便利機能 -----
\usepackage{hyperref}
\usepackage{cleveref}
\crefname{figure}{図}{図}
\crefname{table}{表}{表}
\crefname{section}{第}{第}
\crefname{equation}{式}{式}
\crefname{listing}{リスト}{リスト}

%==============================================================================
% ドキュメント情報
%==============================================================================
\title{船体運動力学 課題 2-5}
\author{地球総合工学科 \quad B3 \quad 08C23031 \quad 古賀 光一朗}
\date{\today}

%==============================================================================
% 本文開始
%==============================================================================
\begin{document}

\maketitle

グリーンの公式を用いれば、下記の通りにディフラクションポテンシャル$\phi_4$による圧力積分は、ラデ
ィエイションポテンシャル$\phi_2$による圧力積分に置き換えられることを説明しなさい
$$
\Delta f_{y(d)} = Re\left[\rho g \zeta_a e^{i \omega t} \iint_{S} e_2 \cdot \phi_4 n dS \right] = - Re \left[\frac{\rho g  \zeta_a}{v_a}e^{i\omega t} \iint_S \phi_2 \frac{\partial \phi_0}{\partial n} dS \right]
$$
\underline{提出期限:2025年 11月 17日(月)10:30AM}

\vspace{10mm}

グリーンの公式
$$
\iint_{S} \left( \phi \frac{\partial \psi}{\partial n} - \psi \frac{\partial \phi}{\partial n}\right)dS = \iiint_V\left( \phi \nabla^2\psi - \psi \nabla^2 \phi \right)dV
$$
ここで、$\phi_2$(ラディエイションポテンシャル)と$\phi_4$(ディフラクションポテンシャル)は、物体表面$S_H$以外(自由表面$S_F$、水底$S_B$、無限遠方$S_\infty$)において同じ境界条件を満たすため、これらの面状での積分は互いにキャンセルされ0となる。
したがって、積分領域は物体表面$S_H$のみとなり、
$$
\iint_{S_H} \left( \phi_4 \frac{\partial \phi_2}{\partial n} - \phi_2 \frac{\partial \phi_4}{\partial n}\right)dS = 0
$$
これを移項すると、以下の関係式が得られる。
$$
\iint_{S_H} \phi_4 \frac{\partial \phi_2}{\partial n} dS = \iint_{S_H} \phi_2 \frac{\partial \phi_4}{\partial n} dS
$$

\vspace{5mm}

次に、課題のディフラクション力$\Delta f_{y(d)}$の計算式を考える。
$$
\Delta f_{y(d)} = Re\left[\rho g \zeta_a e^{i \omega t} \iint_{S_H} \phi_4 (\mathbf{e}_2 \cdot \mathbf{n}) dS \right]
$$
(課題の式では積分領域が$S$となっているが、ここでは物体表面$S_H$とする。また $e_2 \cdot \phi_4 n dS$ を $\phi_4 (\mathbf{e}_2 \cdot \mathbf{n}) dS$ と解釈した)

ここで、ラディエイション問題(ヒーブ運動)における物体表面$S_H$での境界条件を考える。
ヒーブ方向($y$軸、$\mathbf{e}_2$方向)の速度ベクトルを$\mathbf{v} = v_a \mathbf{e}_2$とすると、物体表面の法線方向速度$v_n$は、
$$
v_n = \mathbf{v} \cdot \mathbf{n} = (v_a \mathbf{e}_2) \cdot \mathbf{n} = v_a (\mathbf{e}_2 \cdot \mathbf{n})
$$
一方、ラディエイションポテンシャル$\phi_2$の定義より、
$$
v_n = \frac{\partial \phi_2}{\partial n}
$$
したがって、
$$
v_a (\mathbf{e}_2 \cdot \mathbf{n}) = \frac{\partial \phi_2}{\partial n}
$$
が成り立ち、これを変形すると
$$
\mathbf{e}_2 \cdot \mathbf{n} = \frac{1}{v_a} \frac{\partial \phi_2}{\partial n}
$$
となる。

これをディフラクション力の式の積分項に代入すると、
$$
\iint_{S_H} \phi_4 (\mathbf{e}_2 \cdot \mathbf{n}) dS = \iint_{S_H} \phi_4 \left( \frac{1}{v_a} \frac{\partial \phi_2}{\partial n} \right) dS = \frac{1}{v_a} \iint_{S_H} \phi_4 \frac{\partial \phi_2}{\partial n} dS
$$
先に導出したグリーンの公式による関係式 $\iint_{S_H} \phi_4 \frac{\partial \phi_2}{\partial n} dS = \iint_{S_H} \phi_2 \frac{\partial \phi_4}{\partial n} dS$ を用いて、上式を書き換えると、
$$
\frac{1}{v_a} \iint_{S_H} \phi_4 \frac{\partial \phi_2}{\partial n} dS = \frac{1}{v_a} \iint_{S_H} \phi_2 \frac{\partial \phi_4}{\partial n} dS
$$
となる。

ここで、ディフラクション問題において、入射波ポテンシャル$\phi_0$とディフラクションポテンシャル$\phi_4$の物体表面での境界条件は、
$$
\frac{\partial (\phi_0 + \phi_4)}{\partial n} = 0 \quad \text{すなわち} \quad \frac{\partial \phi_4}{\partial n} = - \frac{\partial \phi_0}{\partial n}
$$
であるため、これを代入すると、
$$
\frac{1}{v_a} \iint_{S_H} \phi_2 \frac{\partial \phi_4}{\partial n} dS = \frac{1}{v_a} \iint_{S_H} \phi_2 \left( - \frac{\partial \phi_0}{\partial n} \right) dS = - \frac{1}{v_a} \iint_{S_H} \phi_2 \frac{\partial \phi_0}{\partial n} dS
$$
以上をまとめると、
$$
\iint_{S_H} \phi_4 (\mathbf{e}_2 \cdot \mathbf{n}) dS = - \frac{1}{v_a} \iint_{S_H} \phi_2 \frac{\partial \phi_0}{\partial n} dS
$$
が導かれる。

これを元のディフラクション力の式に代入すれば、
\begin{align*}
\Delta f_{y(d)} &= Re\left[\rho g \zeta_a e^{i \omega t} \iint_{S_H} \phi_4 (\mathbf{e}_2 \cdot \mathbf{n}) dS \right] \\
&= Re\left[\rho g \zeta_a e^{i \omega t} \left( - \frac{1}{v_a} \iint_{S_H} \phi_2 \frac{\partial \phi_0}{\partial n} dS \right) \right] \\
&= - Re \left[\frac{\rho g \zeta_a}{v_a}e^{i\omega t} \iint_{S_H} \phi_2 \frac{\partial \phi_0}{\partial n} dS \right]
\end{align*}
となり、課題の式(積分領域を$S_H$とし、$S$と表記)が示された。

\end{document}


\end{document}
