%==============================================================================
% LaTeXの処理に関するおまじない (マジックコメント)
%==============================================================================
% !TEX encoding = UTF-8
%==============================================================================
% ドキュメントクラス
%==============================================================================
\documentclass[11pt]{ltjsarticle}

%==============================================================================
%【必須級】便利なパッケージたち
%==============================================================================

%----- ページ設定 -----
\usepackage[
    top=20mm,
    bottom=20mm,
    left=20mm,
    right=20mm
]{geometry}

%----- フォント設定 -----
\usepackage[T1]{fontenc}
\usepackage{newtxtext}
\usepackage{courier} 
\usepackage{newtxmath}
\usepackage{textcomp}
\usepackage{newtxtt}

%----- 数式関連 -----
\usepackage{amsmath}
\usepackage[detect-all]{siunitx}

%----- 図表・画像関連 -----
\usepackage{graphicx}
\usepackage{here}
\usepackage{booktabs}
\usepackage{float}

%----- ソースコード表示 -----
\usepackage{listings}
\lstset{
    basicstyle=\ttfamily\small,
    breaklines=true,
    frame=single,
    commentstyle={\itshape \color[gray]{0.5}},
    keywordstyle={\bfseries \color{blue}},
    stringstyle={\color{red}},
    showstringspaces=false,
    numbers=left,
    numberstyle=\tiny\color[gray]{0.5},
    captionpos=b
}

%----- その他便利機能 -----
\usepackage{hyperref}
\usepackage{cleveref}
\crefname{figure}{図}{図}
\crefname{table}{表}{表}
\crefname{section}{第}{第}
\crefname{equation}{式}{式}
\crefname{listing}{リスト}{リスト}

%==============================================================================
% ドキュメント情報
%==============================================================================
\title{船体運動力学 課題 2-3}
\author{地球総合工学科 \quad B3 \quad 08C23031 \quad 古賀 光一朗}
\date{\today}

%==============================================================================
% 本文開始
%==============================================================================
\begin{document}

\maketitle

次式で表される進行波の時空間変位
$$
\zeta(x_1, t) = \zeta_a \cos(kx_1 - \omega t)
$$
に対して次の支配方程式と境界条件が与えられている。

\bigskip % 条件間を少しあける

[L]ラプラス方程式
$$
\nabla^2 \Phi = \frac{\partial^2 \Phi}{\partial x_1^2} + \frac{\partial^2 \Phi}{\partial x_3^2} = 0
$$

[F]自由表面条件
$$
\frac{\partial^2 \Phi}{\partial t^2} - g \frac{\partial \Phi}{\partial x_3} = 0 \quad \text{at} \ x_3 = 0
$$

[B]水底条件
$$
\frac{\partial \Phi}{\partial x_3} = 0 \quad \text{at} \ x_3 = h
$$

\bigskip

これより、進行波の速度ポテンシャルを求め(導出過程を整理し)、流体内の進行波による変動圧力を数式で示しなさい。なお、記号の意味や座標系は講義資料に従う。

\vspace{2cm} % 提出期限との間のスペース

% 提出期限
提出期限:2025年 11月 10日 (月) 10:30AM

------------------------------------------------------------------------------------------------------------------------------------------------
\subsection*{1. 速度ポテンシャル $\Phi$ の導出}

速度ポテンシャル $\Phi(x_1, x_3, t)$ を変数分離を用いて求め、
$$
\Phi(x_1, x_3, t) = Z(x_3) \sin(kx_1 - \omega t)
$$
と仮定する。

\subsubsection*{[L] ラプラス方程式への代入}
$\Phi$ を [L] に代入すると、$Z(x_3)$ に関する常微分方程式

$$
Z''(x_3) - k^2 Z(x_3) = 0
$$

が得られる。この一般解は、

$$
Z(x_3) = A_3 \exp(-k x_3) + B_3 \exp(k x_3)
$$

となる。

\subsubsection*{[B] 水底条件の適用}
[B] $\frac{\partial \Phi}{\partial x_3} = 0$ at $x_3 = h$ を適用する。

$$
\frac{\partial \Phi}{\partial x_3} = Z'(x_3) \sin(kx_1 - \omega t)
$$

$$
Z'(x_3) = -k A_3 \exp(-k x_3) + k B_3 \exp(k x_3)
$$

$x_3 = h$ を代入して $Z'(h) = 0$ より、

$$
-k A_3 \exp(-kh) + k B_3 \exp(kh) = 0
$$

係数 $A_3, B_3$ の関係式が得られる。

\begin{equation}
\frac{B_3}{A_3} = \frac{\exp(-kh)}{\exp(kh)} = \exp(-2kh)
\label{eq:B_A_ratio}
\end{equation}

\subsubsection*{[F] 自由表面条件の適用}

[F] $\frac{\partial^2 \Phi}{\partial t^2} - g \frac{\partial \Phi}{\partial x_3} = 0$ at $x_3 = 0$ を適用する。

$$
\frac{\partial^2 \Phi}{\partial t^2} = Z(x_3) \cdot [-\omega^2 \sin(kx_1 - \omega t)]
$$

これらを[F]に代入し、$x_3=0$ とすると、

$$
[-\omega^2 Z(0) - g Z'(0)] \sin(kx_1 - \omega t) = 0
$$

よって、

$$
-\omega^2 Z(0) - g Z'(0) = 0
$$

ここで、$Z(0) = A_3 + B_3$、$Z'(0) = k(B_3 - A_3)$ を代入すると、

$$
-\omega^2 (A_3 + B_3) - g k (B_3 - A_3) = 0
$$

$$
(-\omega^2 + gk) A_3 + (-\omega^2 - gk) B_3 = 0
$$

これも $A_3, B_3$ の関係式である。

\begin{equation}
(gk - \omega^2) A_3 = (gk + \omega^2) B_3
\label{eq:A_B_relation}
\end{equation}

\subsubsection*{分散関係式の導出}

式 \eqref{eq:B_A_ratio} を 式 \eqref{eq:A_B_relation} に代入し、$\frac{B_3}{A_3}$ を消去する。

$$
(gk - \omega^2) = (gk + \omega^2) \frac{B_3}{A_3}
$$

$$
\frac{gk - \omega^2}{gk + \omega^2} = \exp(-2kh) = \frac{\exp(-kh)}{\exp(kh)}
$$

$$
(gk - \omega^2) \exp(kh) = (gk + \omega^2) \exp(-kh)
$$

$\omega^2$ と $k$ の項で整理すると、

$$
gk (\exp(kh) - \exp(-kh)) = \omega^2 (\exp(kh) + \exp(-kh))
$$

$2 \sinh(kh) = \exp(kh) - \exp(-kh)$ および $2 \cosh(kh) = \exp(kh) + \exp(-kh)$ の関係より、

$$
gk (2 \sinh(kh)) = \omega^2 (2 \cosh(kh))
$$

したがって、分散関係式が得られる。

\begin{equation}
\omega^2 = gk \tanh(kh)
\label{eq:dispersion}
\end{equation}

\subsubsection*{$Z(x_3)$ の整理と $A_3$ の決定}

式 \eqref{eq:B_A_ratio} の関係 $B_3 = A_3 \exp(-2kh)$ を $Z(x_3)$ の一般解に代入する。

$$
\begin{aligned}
Z(x_3) &= A_3 \exp(-k x_3) + (A_3 \exp(-2kh)) \exp(k x_3) \\
&= A_3 (\exp(-k x_3) + \exp(k x_3 - 2kh)) \\
&= A_3 \exp(-kh) (\exp(-k x_3 + kh) + \exp(k x_3 - kh)) \\
&= 2 A_3 \exp(-kh) \cosh(k(x_3 - h))
\end{aligned}
$$

よって、速度ポテンシャルは

$$
\Phi(x_1, x_3, t) = 2 A_3 \exp(-kh) \cosh(k(x_3 - h)) \sin(kx_1 - \omega t)
$$

と書ける。
次に、関係式 $\zeta = -\frac{1}{g} \frac{\partial \Phi}{\partial t} \big|_{x_3=0}$ と $\zeta(x_1, t) = \zeta_a \cos(kx_1 - \omega t)$ を用いて $A_3$ を決定する。

$$
\frac{\partial \Phi}{\partial t} \Bigg|_{x_3=0} = -2 \omega A_3 \exp(-kh) \cosh(-kh) \cos(kx_1 - \omega t)
$$

$$
\zeta(x_1, t) = -\frac{1}{g} \left[ -2 \omega A_3 \exp(-kh) \cosh(kh) \cos(kx_1 - \omega t) \right]
$$

$$
\zeta(x_1, t) = \frac{2 \omega A_3}{g} \exp(-kh) \cosh(kh) \cos(kx_1 - \omega t)
$$

$\zeta_a$ と係数を比較して、

$$
\zeta_a = \frac{2 \omega A_3}{g} \exp(-kh) \cosh(kh)
$$

$A_3$ について解くと、

$$
A_3 = \frac{g \zeta_a \exp(kh)}{2 \omega \cosh(kh)}
$$

これを $\Phi$ の式に代入して整理すると、最終的な速度ポテンシャルが得られる。

\begin{equation}
\Phi(x_1, x_3, t) = \frac{g \zeta_a}{\omega} \frac{\cosh(k(x_3 - h))}{\cosh(kh)} \sin(kx_1 - \omega t)
\label{eq:Phi_final}
\end{equation}

\subsection*{2. 流体内の変動圧力 $p_d$ の導出}

微小振幅波理論における、線形化されたベルヌーイの式は

$$
\frac{p}{\rho} + \frac{\partial \Phi}{\partial t} + g x_3 = C \quad (\text{Cは定数})
$$

である。ここで $p$ はゲージ圧であり、静水圧成分 $p_s = -\rho g x_3$ と変動圧力成分 $p_d$ の和 $p = p_s + p_d$ と表せる($x_3$が下向き正のため静水圧は負号)。

$$
\frac{-\rho g x_3 + p_d}{\rho} + \frac{\partial \Phi}{\partial t} + g x_3 = C
$$

$$
-g x_3 + \frac{p_d}{\rho} + \frac{\partial \Phi}{\partial t} + g x_3 = C
$$

変動圧力 $p_d$ は、流れがない状態 ($\Phi=0$) で $p_d=0$ となるため、定数 $C=0$ とおける。

\begin{equation}
p_d = -\rho \frac{\partial \Phi}{\partial t}
\label{eq:pd_def}
\end{equation}

式 \eqref{eq:Phi_final} の $\Phi$ を時間 $t$ で偏微分する。

$$
\begin{aligned}
\frac{\partial \Phi}{\partial t} &= \frac{\partial}{\partial t} \left[ \frac{g \zeta_a}{\omega} \frac{\cosh(k(x_3 - h))}{\cosh(kh)} \sin(kx_1 - \omega t) \right] \\
&= \frac{g \zeta_a}{\omega} \frac{\cosh(k(x_3 - h))}{\cosh(kh)} \cdot [-\omega \cos(kx_1 - \omega t)] \\
&= -g \zeta_a \frac{\cosh(k(x_3 - h))}{\cosh(kh)} \cos(kx_1 - \omega t)
\end{aligned}
$$

これを式 \eqref{eq:pd_def} に代入して、変動圧力 $p_d$ を得る。

\begin{equation}
p_d(x_1, x_3, t) = \rho g \zeta_a \frac{\cosh(k(x_3 - h))}{\cosh(kh)} \cos(kx_1 - \omega t)
\label{eq:pd_final}
\end{equation}

\end{document}
