%==============================================================================
% LaTeXの処理に関するおまじない (マジックコメント)
%==============================================================================
% !TEX encoding = UTF-8
%==============================================================================
% ドキュメントクラス
%==============================================================================
\documentclass[11pt]{ltjsarticle}

%==============================================================================
%【必須級】便利なパッケージたち
%==============================================================================

%----- ページ設定 -----
\usepackage[
    top=20mm,
    bottom=20mm,
    left=20mm,
    right=20mm
]{geometry}

%----- フォント設定 -----
\usepackage[T1]{fontenc}
\usepackage{newtxtext}
\usepackage{courier} 
\usepackage{newtxmath}
\usepackage{textcomp}
\usepackage{newtxtt}

%----- 数式関連 -----
\usepackage{amsmath}
\usepackage[detect-all]{siunitx}

%----- 図表・画像関連 -----
\usepackage{graphicx}
\usepackage{here}
\usepackage{booktabs}
\usepackage{float}

%----- ソースコード表示 -----
\usepackage{listings}
\lstset{
    basicstyle=\ttfamily\small,
    breaklines=true,
    frame=single,
    commentstyle={\itshape \color[gray]{0.5}},
    keywordstyle={\bfseries \color{blue}},
    stringstyle={\color{red}},
    showstringspaces=false,
    numbers=left,
    numberstyle=\tiny\color[gray]{0.5},
    captionpos=b
}

%----- その他便利機能 -----
\usepackage{hyperref}
\usepackage{cleveref}
\crefname{figure}{図}{図}
\crefname{table}{表}{表}
\crefname{section}{第}{第}
\crefname{equation}{式}{式}
\crefname{listing}{リスト}{リスト}

%==============================================================================
% ドキュメント情報
%==============================================================================
\title{構造荷重論 レポート課題1}
\author{地球総合工学科 \quad B3 \quad 08C23031 \quad 古賀 光一朗}
\date{\today}

%==============================================================================
% 本文開始
%==============================================================================
\begin{document}

\maketitle

\subsection*{1}
波浪中(波向きは船長方向と一致)の船体中央での縦曲げモーメントの位相ゼロ成分は、
\begin{equation}
B.M. = \rho g h_A B \left(\frac{L}{2}\right)^2 \frac{1}{\gamma^2} \left[ \frac{1}{2} \gamma \sin\gamma - 1 + \cos\gamma \right] \quad (*)
\end{equation}

の式で与えられた。記号などについては授業中の資料を参照のこと。

\begin{itemize}
    \item(1) 波長が無限大(もしくは周波数がゼロ)に漸近するとき、縦曲げモーメントはどのような値に収束するか、式(*)より導きなさい。
    \item(2) このことが力学的に正しいことを説明しなさい。
\end{itemize}


(1)


$\lambda$
が無限大に漸近するとき、$\gamma$は$0$に漸近する。したがって、
(*)式は
\begin{align}
    B.M. &= \rho g h_A B \left(\frac{L}{2}\right)^2 (\lim_{\gamma \to 0} \frac{1}{\gamma^2} [ \frac{1}{2} \gamma \sin\gamma - 1 + \cos\gamma ]) \\
         &= \rho g h_A B \left(\frac{L}{2}\right)^2 (\lim_{\gamma \to 0} \{\frac{\sin\gamma}{2\gamma}+ \frac{\cos\gamma - 1}{\gamma^2 })\} \\
&= 0
  \end{align}


(2)


波長が無限大に漸近するということは、波の山と谷の間隔が非常に長くなることを意味する。したがって、船体全体がほぼ同じ高さに位置することになり、船体にかかる波の影響が均一化される。このため、船体中央での縦曲げモーメントはゼロに近づく。力学的には、波の影響が均一化されることで、船体にかかる応力分布が均等化され、結果として縦曲げモーメントが減少することが理解できる。

\subsection*{2}
位相 $\pi/2$ ($\omega t = \pi/2$, $\sin$ 成分) の縦曲げモーメントは以下の式となることを説明した。

\begin{itemize}
    \item[(1)] この導出を示した部分に誤りの表記があることが判明した。配布テキストの誤りを指摘しなさい(係数について大きく二か所の誤りがある)。
\end{itemize}

\begin{align*}
  B.M. = & \frac{3}{2} \frac{\rho g b h_A }{\left(\frac{kL}{2}\right)^2}\left(\frac{L}{2}\right)^2 \left(-\frac{kL}{2} \cos\frac{kL}{2} + \sin\frac{kL}{2}\right) \left(\left(1-\left(\frac{x}{\frac{L}{2}}\right)\right)-\frac{1}{3}\left(1-\left(\frac{x}{\frac{L}{2}}\right)^3\right)\right)\\
         & -\frac{\rho g b h_A }{\left(\frac{kL}{2}\right)^2}\left(\frac{L}{2}\right)^2 \left(\left(\frac{kL}{2} \cos\frac{kL}{2} - \sin\frac{kL}{2}\right) - \left(kx\cos{\frac{kL}{2} - \sin{kx}}\right)\right) \\
\end{align*}



(2)



\begin{align*}
  EI\frac{\partial^4 w(x,t)}{\partial x^4} = & - \frac{\rho g B h_A \frac{L^2}{2}\left(\frac{kL}{2}\right)^{-2}\left(-\frac{kL}{2}\cos{\frac{kL}{2}} + \sin{\frac{kL}{2}}\right)}{\frac{L^3}{12}}x + \rho g B h_A \sin{kx} \\
  = & - \frac{\rho g B h_A \left(\frac{kL}{2}\right)^{-2}\left(-\frac{kL}{2}\cos{\frac{kL}{2}} + \sin{\frac{kL}{2}}\right)}{L}x + \rho g B h_A \sin{kx} \\
\end{align*}

とあるが、この次の式変形に誤りを発見した。

一階積分を行うと、

\begin{align*}
        SF(L/2) - EI\frac{\partial^3 w(x,t)}{\partial x^3}
        = & -\rho g B h_A \frac{3}{2\left(\frac{L}{2}\right)} \left(\frac{(L/2)^2}{(L/2)^2}\right) \left(\frac{kL}{2}\right)^{-2}\left(-\frac{kL}{2}\cos{\frac{kL}{2}} + \sin{\frac{kL}{2}}\right)\left(\left(\frac{L}{2}\right
     )^2-x^2\right) \\
          & - \frac{\rho g B h_A }{k}\left(\cos{\frac{kL}{2}} - \cos{kx}\right) \\
        = & -\rho g B h_A \frac{L}{8} \left(\frac{kL}{2}\right)^{-2}\left(-\frac{kL}{2}\cos{\frac{kL}{2}} + \sin{\frac{kL}{2}}\right) \left( 1-\left(\frac{x}{L/2}\right)^2\right) \\
          & - \frac{\rho g B h_A}{\frac{kL}{2}}\frac{L}{2}\left(\cos{\frac{kL}{2}} - \cos{kx}\right)
\end{align*}

と記載れているが、正しくは


\begin{align*}
        SF(L/2) - EI\frac{\partial^3 w(x,t)}{\partial x^3}
        = & \rho g B h_A \frac{3}{2\left(\frac{L}{2}\right)} \left(\frac{(L/2)^2}{(L/2)^2}\right) \left(\frac{kL}{2}\right)^{-2}\left(-\frac{kL}{2}\cos{\frac{kL}{2}} + \sin{\frac{kL}{2}}\right)\left(\left(\frac{L}{2}\right
     )^2-x^2\right) \\
          &  \frac{\rho g B h_A }{k}\left(\cos{\frac{kL}{2}} - \cos{kx}\right) \\
        = & \rho g B h_A \frac{3L}{4} \left(\frac{kL}{2}\right)^{-2}\left(-\frac{kL}{2}\cos{\frac{kL}{2}} + \sin{\frac{kL}{2}}\right) \left( 1-\left(\frac{x}{L/2}\right)^2\right) \\
          &  \frac{\rho g B h_A}{\frac{kL}{2}}\frac{L}{2}\left(\cos{\frac{kL}{2}} - \cos{kx}\right)
\end{align*}


\begin{itemize}
      \item[(2)] (2)上の式で、両端でのモーメントがゼロになることを示しなさい。また、力学的になぜ両端でモーメントがゼロになるかを説明しなさい。
\end{itemize}


(2)


両端 $x = \pm L/2$ でゼロになることを示す。

\subsubsection*{(i) $x = L/2$ (船尾側) の場合}
$x = L/2$ を代入する。
このとき、$\frac{x}{L/2} = 1$、 $kx = \frac{kL}{2}$

第1項の $x$ に依存する部分は、
$$
\left(1-\frac{x}{L/2}\right)-\frac{1}{3}\left(1-\left(\frac{x}{L/2}\right)^3\right) = (1-1) - \frac{1}{3}(1 - 1^3) = 0 - 0 = 0
$$
となり、第1項全体が $0$ となる。

第2項の $x$ に依存する部分は、
$$
\left(kx\cos{\frac{kL}{2}} - \sin{kx}\right) = \left(\frac{kL}{2}\cos{\frac{kL}{2}} - \sin{\frac{kL}{2}}\right)
$$
となるため、第2項全体は、
$$
-\frac{\rho g B h_A }{\left(\frac{kL}{2}\right)^2}\left(\frac{L}{2}\right)^2 \left[ \left(\frac{kL}{2} \cos\frac{kL}{2} - \sin\frac{kL}{2}\right) - \left(\frac{kL}{2}\cos{\frac{kL}{2}} - \sin{\frac{kL}{2}}\right) \right] = 0
$$
したがって、$B.M.(x=L/2) = 0 + 0 = 0$ 

\subsubsection*{(ii) $x = -L/2$ (船首側) の場合}
$x = -L/2$ を代入する
このとき、$\frac{x}{L/2} = -1$、 $kx = -\frac{kL}{2}$ となります。$\sin$ は奇関数であるため $\sin(kx) = \sin(-\frac{kL}{2}) = -\sin(\frac{kL}{2})$ 

第1項の $x$ に依存する部分は、
\begin{align*}
\left(1-\frac{x}{L/2}\right)-\frac{1}{3}\left(1-\left(\frac{x}{L/2}\right)^3\right) &= (1 - (-1)) - \frac{1}{3}(1 - (-1)^3) \\
&= 2 - \frac{1}{3}(1 - (-1)) \\
&= 2 - \frac{2}{3} = \frac{4}{3}
\end{align*}
よって、第1項は $T_1$ とおくと、
\begin{align*}
T_1 &= -\frac{3}{2} \frac{\rho g B h_A }{\left(\frac{kL}{2}\right)^2}\left(\frac{L}{2}\right)^2 \left(-\frac{kL}{2} \cos\frac{kL}{2} + \sin\frac{kL}{2}\right) \times \frac{4}{3} \\
&= -2 \frac{\rho g B h_A }{\left(\frac{kL}{2}\right)^2}\left(\frac{L}{2}\right)^2 \left(-\frac{kL}{2} \cos\frac{kL}{2} + \sin\frac{kL}{2}\right)
\end{align*}

第2項の $x$ に依存する部分は、
\begin{align*}
\left(kx\cos{\frac{kL}{2}} - \sin{kx}\right) &= \left(-\frac{kL}{2}\cos{\frac{kL}{2}} - \sin(-\frac{kL}{2})\right) \\
&= \left(-\frac{kL}{2}\cos{\frac{kL}{2}} + \sin{\frac{kL}{2}}\right)
\end{align*}
よって、第2項は $T_2$ とおくと、
\begin{align*}
T_2 &= -\frac{\rho g B h_A }{\left(\frac{kL}{2}\right)^2}\left(\frac{L}{2}\right)^2 \left[ \left(\frac{kL}{2} \cos\frac{kL}{2} - \sin\frac{kL}{2}\right) - \left(-\frac{kL}{2}\cos{\frac{kL}{2}} + \sin{\frac{kL}{2}}\right) \right] \\
&= -\frac{\rho g B h_A }{\left(\frac{kL}{2}\right)^2}\left(\frac{L}{2}\right)^2 \left[ 2\left(\frac{kL}{2} \cos\frac{kL}{2} - \sin\frac{kL}{2}\right) \right] \\
&= +2 \frac{\rho g B h_A }{\left(\frac{kL}{2}\right)^2}\left(\frac{L}{2}\right)^2 \left[ -\frac{kL}{2} \cos\frac{kL}{2} + \sin\frac{kL}{2} \right]
\end{align*}

$T_1$ と $T_2$ は、大きさが同じで符号が逆であるため、
$B.M.(x=-L/2) = T_1 + T_2 = 0$ となる。
以上より、両端 $x = \pm L/2$ でモーメントがゼロになる

\subsubsection*{力学的な理由}
船体は水に浮かんでおり、その両端(船首・船尾)は壁などに固定されていない「自由端」である。自由端は外部からの拘束を受けないため、曲げモーメントやせん断力が $0$ になるという境界条件が適用できる。
したがって、船体梁の縦曲げモーメントは、両端 $x = \pm L/2$ において $0$ とならなければならない。

\end{document}
