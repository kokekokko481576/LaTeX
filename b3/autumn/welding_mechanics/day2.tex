\documentclass[10.5pt, a4paper]{article}

% --- パッケージ設定 ---
\usepackage[margin=20mm]{geometry} % マージンを上下左右20mmに設定
\usepackage{newtxtext, newtxmath}     % フォントをTimes New Roman系に設定
\usepackage{amsmath}               % 数式用パッケージ
\usepackage{graphicx}              % 図の挿入用パッケージ
\usepackage{setspace}              % 行間設定用
\usepackage{url}                   % URLをきれいに改行するためのパッケージを追加
\usepackage{hyperref}              % URLをクリッカブルにするパッケージ
\linespread{1.1}                   % 適度な行間を設定

% --- ヘッダー・フッターを消去 ---
\pagestyle{empty}

\begin{document}

% --- タイトルと氏名、学籍番号 ---
\begin{center}
    \Large{\textbf{溶接力学 第2回宿題}}
\end{center}

\vspace{1cm} % 1cmの垂直方向の空白


\noindent 氏名: 古賀光一朗 \\ 
学籍番号: 08C23031 \\
提出日: 2025/10/10
\vspace{1.5cm} % 1.5cmの垂直方向の空白
\section{下記、軟鋼の材料特性を用いて拘束棒の降伏応力$T_Y$を計算してください。}
\begin{itemize}
    \item ヤング率: $E = 2\times10^{5}$[MPa]
    \item 熱膨張係数: $\alpha = 1\times10^{-5}$[1/℃]
    \item 降伏応力$\sigma_Y = 200$[MPa]
\end{itemize}
$$T_Y = \frac{200 \times10^6}{2 \times 10^5 \times 10^6 \times 10^{-5}}=100\text{℃}$$
\section{下記、高張力鋼の材料特性を用いて拘束棒の降伏応力$T_Y$を計算してください。}
\begin{itemize}
    \item ヤング率: $E = 2\times10^{5}$[MPa]
    \item 熱膨張係数: $\alpha = 1.2 \times10^{-5}$[1/℃]
    \item 降伏応力$\sigma_Y = 480$[MPa]
\end{itemize}
$$T_Y = \frac{480 \times10^6}{2 \times 10^5 \times 10^6 \times 1.2\times 10^{-5}}=200\text{℃}$$
\section{下記、ステンレスの材料特性を用いて拘束棒の降伏応力$T_Y$を計算してください。}
\begin{itemize}
    \item ヤング率: $E = 2\times10^{5}$[MPa]
    \item 熱膨張係数: $\alpha = 1.8\times10^{-5}$[1/℃]
    \item 降伏応力$\sigma_Y = 360$[MPa]
\end{itemize}
$$T_Y = \frac{360 \times10^6}{2 \times 10^5 \times 10^6 \times 1.8 \times 10^{-5}}=100\text{℃}$$
\section{下記、アルミ合金の材料特性を用いて拘束棒の降伏応力$T_Y$を計算してください。}
\begin{itemize}
    \item ヤング率: $E = 7\times10^{4}$[MPa]
    \item 熱膨張係数: $\alpha = 2\times10^{-5}$[1/℃]
    \item 降伏応力$\sigma_Y = 140$[MPa]
\end{itemize}
$$T_Y = \frac{140 \times10^6}{7 \times 10^4 \times 10^6 \times 2 \times 10^{-5}}=100\text{℃}$$
\end{document}