%==============================================================================
% LaTeXの処理に関するおまじない (マジックコメント)
%==============================================================================
% !TEX encoding = UTF-8
%==============================================================================
% ドキュメントクラス
%==============================================================================
\documentclass[11pt]{ltjsarticle}

%==============================================================================
%【必須級】便利なパッケージたち
%==============================================================================

%----- ページ設定 -----
\usepackage[
    top=20mm,
    bottom=20mm,
    left=20mm,
    right=20mm
]{geometry}

%----- フォント設定 -----
\usepackage[T1]{fontenc}
\usepackage{newtxtext}
\usepackage{courier} 
\usepackage{newtxmath}
\usepackage{textcomp}
\usepackage{newtxtt}

%----- 数式関連 -----
\usepackage{amsmath}
\usepackage[detect-all]{siunitx}

%----- 図表・画像関連 -----
\usepackage{graphicx}
\usepackage{here}
\usepackage{booktabs}
\usepackage{float}

%----- ソースコード表示 -----
\usepackage{listings}
\lstset{
    basicstyle=\ttfamily\small,
    breaklines=true,
    frame=single,
    commentstyle={\itshape \color[gray]{0.5}},
    keywordstyle={\bfseries \color{blue}},
    stringstyle={\color{red}},
    showstringspaces=false,
    numbers=left,
    numberstyle=\tiny\color[gray]{0.5},
    captionpos=b
}

%----- その他便利機能 -----
\usepackage{hyperref}
\usepackage{cleveref}
\crefname{figure}{図}{図}
\crefname{table}{表}{表}
\crefname{section}{第}{第}
\crefname{equation}{式}{式}
\crefname{listing}{リスト}{リスト}

%==============================================================================
% ドキュメント情報
%==============================================================================
\title{溶接力学第3回課題}
\author{地球総合工学科 \quad B3 \quad 08C23031 \quad 古賀 光一朗}
\date{\today}

%==============================================================================
% 本文開始
%==============================================================================

\begin{document}

\maketitle

\section{A1温度とA3温度}
\textbf{鋼材の$A_1$温度と$A_3$温度で、何が起こっていますか?}

鋼材の加熱・冷却時に組織が変化する変態点である。

\begin{itemize}
    \item \textbf{A1温度(共析変態点)}
    \begin{itemize}
        \item 鋼が加熱される際、パーライト組織(フェライト+セメンタイト)が\textbf{オーステナイト}に変態(共析変態)を開始する温度である。
    \end{itemize}
    \item \textbf{A3温度(変態点)}: 炭素量に依存
    \begin{itemize}
        \item 亜共析鋼において、加熱時に\textbf{フェライト}から\textbf{オーステナイト}への変態が完了する温度である。$A_1$~$A_3$温度間は、フェライトとオーステナイトの共存領域となる。
    \end{itemize}
\end{itemize}

\section{アーク溶接とレーザー溶接の熱効率}
\textbf{アーク溶接とレーザー溶接の熱効率$\eta$は大体どのぐらいですか?}

\begin{itemize}
    \item \textbf{アーク溶接}: $\eta \approx \SIrange{60}{90}{\percent}$
    \item \textbf{レーザー溶接}: $\eta \approx \SIrange{20}{40}{\percent}$
\end{itemize}

\section{溶接熱源モデル}
\textbf{熱伝導の数値解析に用いる溶接熱源モデルを三つほど回答してね}

\begin{enumerate}
    \item \textbf{Goldakモデル}
    \begin{itemize}
      \item $q_v = \frac{6\sqrt{3}FQ}{\pi\sqrt{\pi}abc} \exp \left( -\left( \frac{3x^2}{a^2} + \frac{3y^2}{b^2} + \frac{3z^2}{c^2} \right) \right)$ 
    \end{itemize}
    \item \textbf{ガウス熱源モデル}
    \begin{itemize}
      \item $q_v = \frac{3Q}{2\pi R^3}exp[\frac{-3\left(x^2 + y^2 + z^2\right)}{R^2}]$
    \end{itemize}
      \item \textbf{ガウス分布熱源モデル}
    \begin{itemize}
      \item $q_s(x,y) = f_s \frac{3Q}{\pi R^2}exp[\frac{-3\left(x^2 + y^2\right)}{R^2}]$
    \end{itemize}
\end{enumerate}
\end{document}


\section{溶接温度場の最高到達温度}
\textbf{溶接温度場の最高到達温度$T_{max}$は、なんですか?}

溶接部の最高到達温度($T_{\max}$)は、材料の\textbf{沸点 (Boiling Point)}である。

\begin{itemize}
    \item アークやレーザービームが照射される中心部では、金属が溶融する融点 (Melting Point) をはるかに超え、沸点に達して気化(蒸発)する。
\end{itemize}

\section{数値解析結果のチェック項目}
\textbf{数値解析で溶接温度場が得られた場合、何をチェックすべきでしょうか?}


\begin{enumerate}
    \item \textbf{実験結果との比較(妥当性検証)}
    \begin{itemize}
        \item \textbf{熱履歴}: 熱電対などで実測した特定の点の温度履歴(サーマルサイクル)と、解析結果が一致するか。
        \item \textbf{溶融部・ビード形状}: 実際の溶接断面(マクロ組織)と、解析における溶融池の幅や深さ、ビード形状が一致するか。
        \item \textbf{熱影響部(HAZ)の範囲}: 実際の熱影響部 (Heat Affected Zone, HAZ) の大きさと、解析で$A_1$変態点以上に達した領域が一致するか。
    \end{itemize}
    \item \textbf{メッシュ依存性の確認}
    \begin{itemize}
        \item 解析モデルのメッシュ(計算格子)の粗さを変更(細分化)しても、計算結果(温度やHAZ範囲など)が大きく変わらないか(収束しているか)を確認する。
    \end{itemize}
    \item \textbf{物理的な妥当性}
    \begin{itemize}
        \item 最高温度が沸点近傍に収まるなど、物理的に妥当な結果が得られているかを確認する。
    \end{itemize}
\end{enumerate}

\
