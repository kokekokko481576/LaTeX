%==============================================================================
% LaTeXの処理に関するおまじない (マジックコメント)
%==============================================================================
% !TEX encoding = UTF-8
%==============================================================================
% ドキュメントクラス
%==============================================================================
\documentclass[11pt]{ltjsarticle}

%==============================================================================
%【必須級】便利なパッケージたち
%==============================================================================

%----- ページ設定 -----
\usepackage[
    top=20mm,
    bottom=20mm,
    left=20mm,
    right=20mm
]{geometry}

%----- フォント設定 -----
\usepackage[T1]{fontenc}
\usepackage{newtxtext}
\usepackage{courier} 
\usepackage{newtxmath}
\usepackage{textcomp}
\usepackage{newtxtt}

%----- 数式関連 -----
\usepackage{amsmath}
\usepackage[detect-all]{siunitx}

%----- 図表・画像関連 -----
\usepackage{graphicx}
\usepackage{here}
\usepackage{booktabs}
\usepackage{float}

%----- ソースコード表示 -----
\usepackage{listings}
\lstset{
    basicstyle=\ttfamily\small,
    breaklines=true,
    frame=single,
    commentstyle={\itshape \color[gray]{0.5}},
    keywordstyle={\bfseries \color{blue}},
    stringstyle={\color{red}},
    showstringspaces=false,
    numbers=left,
    numberstyle=\tiny\color[gray]{0.5},
    captionpos=b
}

%----- その他便利機能 -----
\usepackage{hyperref}
\usepackage{cleveref}
\crefname{figure}{図}{図}
\crefname{table}{表}{表}
\crefname{section}{第}{第}
\crefname{equation}{式}{式}
\crefname{listing}{リスト}{リスト}

%==============================================================================
% ドキュメント情報
%==============================================================================
\title{溶接力学第2回課題}
\author{地球総合工学科 \quad B3 \quad 08C23031 \quad 古賀 光一朗}
\date{\today}

%==============================================================================
% 本文開始
%==============================================================================

\begin{document}

\maketitle

\section{残留応力分布の正誤と理由}

残留応力は、外力が作用していない部材内部で自己平衡している応力である。したがって、任意の仮想断面において、断面に作用する垂直応力の合力 $F$ とモーメント $M$ は、ともにゼロでなければならない。

$$
F = \int_A \sigma dA = 0
$$

$$
M = \int_A y \sigma dA = 0 \quad (\text{または} \int_A z \sigma dA = 0)
$$

この2つの釣り合い条件に基づき、各分布の妥当性を評価する。

\subsection{幅方向yに分布する残留応力$\sigma_x$ (a), (b)}

この場合、板厚を $t$ とすると、釣り合い条件は以下のようになる。

$$
F_x = t \int \sigma_x(y) dy = 0
$$

$$
M_z = t \int y \sigma_x(y) dy = 0
$$

\begin{itemize}
    \item \textbf{形状 (a): 正しい} \\
    \textbf{理由:} この分布は $y=0$ (溶接部中心) に対して\textbf{対称(偶関数)}である。
    \begin{itemize}
        \item \textbf{合力の釣り合い:} 溶接部の引張応力($+$ 領域)と母材部の圧縮応力($-$ 領域)の面積の総和がゼロになるようバランスさせることが可能である。
        \item \textbf{モーメントの釣り合い:} 被積分関数 $y \sigma_x(y)$ は、奇関数 $y$ と偶関数 $\sigma_x(y)$ の積であるため、\textbf{奇関数}となる。奇関数を $y$ の対称区間(例: $-W/2$ から $W/2$)で積分すると必ずゼロになるため、$M_z = 0$ が自動的に満たされる。
    \end{itemize}

    \item \textbf{形状 (b): 誤り} \\
    \textbf{理由:} 応力 $\sigma_x$ が全域で引張($+$)またはゼロである。
    \begin{itemize}
        \item \textbf{合力の釣り合い:} 積分値 $F_x = t \int \sigma_x(y) dy$ は必ず正の値となり、ゼロにならない。これは合力の釣り合い条件に違反する。
    \end{itemize}
\end{itemize}

\subsection{板厚方向zに分布する残留応力$\sigma_y$ (c), (d)}

この場合、板幅を $W$ とすると、釣り合い条件は以下のようになる。

$$
F_y = W \int \sigma_y(z) dz = 0
$$

$$
M_y = W \int z \sigma_y(z) dz = 0
$$

\begin{itemize}
    \item \textbf{形状 (c): 誤り} \\
    \textbf{理由:} この分布は $z=0$ (板厚中心) に対して\textbf{非対称(奇関数)}である。
    \begin{itemize}
        \item \textbf{合力の釣り合い:} $\sigma_y(z)$ は奇関数であるため、対称区間($-t/2$ から $t/2$)で積分すると $F_y = 0$ となり、合力の釣り合いは満たす。
        \item \textbf{モーメントの釣り合い:} 被積分関数 $z \sigma_y(z)$ は、奇関数 $z$ と奇関数 $\sigma_y(z)$ の積であるため、\textbf{偶関数}となる。偶関数(この場合、常に正または常に負)を対称区間で積分すると、積分値 $M_y$ はゼロにならない。これはモーメントの釣り合い条件に違反する。
    \end{itemize}

    \item \textbf{形状 (d): 正しい} \\
    \textbf{理由:} この分布は $z=0$ (板厚中心) に対して\textbf{対称(偶関数)}である。
    \begin{itemize}
        \item \textbf{合力の釣り合い:} 板厚中央部の圧縮応力($-$ 領域)と、両表面近傍の引張応力($+$ 領域)の面積の総和がゼロになるようバランスさせることが可能である。
        \item \textbf{モーメントの釣り合い:} 被積分関数 $z \sigma_y(z)$ は、奇関数 $z$ と偶関数 $\sigma_y(z)$ の積であるため、\textbf{奇関数}となる。奇関数を $z$ の対称区間($-t/2$ から $t/2$)で積分すると必ずゼロになるため、$M_y = 0$ が自動的に満たされる。
    \end{itemize}
\end{itemize}



\end{document}
