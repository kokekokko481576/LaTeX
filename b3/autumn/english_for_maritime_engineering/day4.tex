%==============================================================================
% LaTeX Magic Comments
%==============================================================================
% !TEX encoding = UTF-8
% !TEX ts-program = platex

%==============================================================================
% Document Class (ユーザーの環境に合わせてjsarticleを使用)
%==============================================================================
\documentclass[11pt, dvipdfmx]{jsarticle}

%==============================================================================
% Packages (ユーザーの元の環境設定をベースにする)
%==============================================================================

%----- ページ設定 -----
\usepackage[
    top=20mm,
    bottom=20mm,
    left=20mm,
    right=20mm
]{geometry}

%----- フォント設定 -----
\usepackage[T1]{fontenc}
\usepackage{newtxtext}
\usepackage{newtxmath}
\usepackage{newtxtt} % URL表示のエラー対策

%----- 数式関連 -----
\usepackage{amsmath}

%----- 図表・画像関連 -----
\usepackage{graphicx}
\usepackage{here}
\usepackage{float}

%----- その他 -----
\usepackage{setspace} % <<<【追加】行間調整のために追加
\usepackage[hidelinks, unicode]{hyperref} % hidelinksでURLの枠線を消す
\usepackage{pxjahyper}
\usepackage{url}

%==============================================================================
% Main Document Start
%==============================================================================
\begin{document}

% --- ヘッダー・フッターを消去 ---
\pagestyle{empty}

% --- フォントサイズを10.5ptに、行間を少し広めに設定 ---
\fontsize{10.5pt}{12.5pt}\selectfont
\linespread{1.09}

% --- タイトルと氏名、学籍番号 ---
\begin{center}
    \Large{\textbf{Terminology Explanation}}
\end{center}

\noindent Name: Koichiro Koga \\
Student ID: 08C23031


% --- 用語 ---
\noindent\large{\textbf{Terminology (Technical word)}}
\hrule
\vspace{0.2cm}
\noindent Linear


% --- 説明 ---
\noindent\large{\textbf{Explanation}}
\hrule
\vspace{0.2cm}
In mathematics and engineering, a relationship is described as \textbf{linear} if it can be represented graphically as a straight line. 
This implies a proportional relationship ($y = ax + b$) between an input and an output, making the system's behavior simple and predictable. 
While most real-world phenomena are governed by complex non-linear dynamics, 
their behavior can often be effectively modeled as linear within a specific, limited operational range. 
This powerful technique, known as \textbf{linear approximation}, is a cornerstone of engineering analysis, 
allowing us to transform potentially unsolvable problems into manageable calculations.
This principle was crucial during the final tuning of our basketball launcher for the NHK Student Robocon. 
To achieve consistent accuracy, we needed a reliable model linking roller speed to shooting distance. 
We conducted a series of controlled experiments, systematically varying the roller speed and measuring the resulting distance for each trial. 
When this empirical data was plotted on a graph, it revealed a surprisingly clear trend: the data points formed an almost perfect straight line, 
indicating the system behaved linearly within our range of interest.
Capitalizing on this discovery, we fit an optimal approximate line to our data points, 
which yielded a simple linear equation that served as an accurate predictive model. With this tool, 
we could instantly calculate the precise motor speed needed for any target distance, 
bypassing the slow and often frustrating process of manual trial-and-error. 
This mathematical shortcut not only saved invaluable time during our final practice sessions 
but also gave us deep confidence in our machine's calibration and consistency.
Our experience is a clear testament to the power of linear approximation in a practical setting. 
It shows how a fundamental mathematical concept can be applied to effectively model, 
predict, and control a complex physical system. This ability to simplify reality into a predictive model is foundational in engineering, 
enabling intelligent, data-driven solutions to replace inefficient guesswork with calculated precision.

\vspace{1cm}

% --- 出典 ---
\noindent\large{\textbf{Source}}
\hrule
\vspace{0.3cm}
\begin{itemize}
    \item "Linear Differential Equations (線形微分方程式)." \textit{Kogakuin University}. Accessed: Oct. 15, 2025. \url{https://brain.cc.kogakuin.ac.jp/~kanamaru/lecture/simdif/03-01.html}
\end{itemize}

\end{document}