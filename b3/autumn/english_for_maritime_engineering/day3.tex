%==============================================================================
% LaTeX Magic Comments
%==============================================================================
% !TEX encoding = UTF-8
% !TEX ts-program = platex

%==============================================================================
% Document Class (ユーザーの環境に合わせてjsarticleを使用)
%==============================================================================
\documentclass[11pt, dvipdfmx]{jsarticle}

%==============================================================================
% Packages (ユーザーの元の環境設定をベースにする)
%==============================================================================

%----- ページ設定 -----
\usepackage[
    top=20mm,
    bottom=20mm,
    left=20mm,
    right=20mm
]{geometry}

%----- フォント設定 -----
\usepackage[T1]{fontenc}
\usepackage{newtxtext}
\usepackage{newtxmath}
\usepackage{newtxtt} % URL表示のエラー対策

%----- 数式関連 -----
\usepackage{amsmath}

%----- 図表・画像関連 -----
\usepackage{graphicx}
\usepackage{here}
\usepackage{float}

%----- その他 -----
\usepackage{setspace} % <<<【追加】行間調整のために追加
\usepackage[hidelinks, unicode]{hyperref} % hidelinksでURLの枠線を消す
\usepackage{pxjahyper}
\usepackage{url}

%==============================================================================
% Main Document Start
%==============================================================================
\begin{document}

% --- ヘッダー・フッターを消去 ---
\pagestyle{empty}

% --- フォントサイズを10.5ptに、行間を少し広めに設定 ---
\fontsize{10.5pt}{12.5pt}\selectfont
\linespread{1.1}

% --- タイトルと氏名、学籍番号 ---
\begin{center}
    \Large{\textbf{Terminology Explanation}}
\end{center}

\vspace{0.2cm}

\noindent Name: Koichiro Koga \\
Student ID: 08C23031


% --- 用語 ---
\noindent\large{\textbf{Terminology (Technical word)}}
\hrule
\vspace{0.1cm}
\noindent Moment of Inertia

% --- 説明 ---
\noindent\large{\textbf{Explanation}}
\hrule
\vspace{0.3cm}
The \textbf{moment of inertia} is a physical quantity that measures an object's resistance to a change in its rotational motion. 
It is the rotational analog of mass; just as an object with a larger mass is harder to accelerate, 
an object with a larger moment of inertia is more difficult to start or stop rotating. 
The key factor is not just the mass itself, but its distribution relative to the axis of rotation. 
For a rigid body, this is calculated by the integral $I = \int r^2 dm$, 
which shows that mass distributed farther from the center contributes more significantly.

A powerful real-world application of this principle was demonstrated in our project for the NHK Student Robocon 2025, 
which required launching a heavy basketball. Our initial design featured a shooter with two small, lightweight rollers. 
However, this prototype failed to perform; the shots were weak and inconsistent. 
The problem lay in the rollers' low moment of inertia. 
When the heavy ball came into contact with the fast-spinning rollers, their rotational speed dropped drastically. 
They could not transfer sufficient kinetic energy because they lacked the necessary stored rotational energy.
The solution was to increase the moment of inertia of the rollers. 
We re-engineered them using a material with a higher mass and a significantly larger diameter. 
According to the relationship $I \propto mr^2$, both changes dramatically increased the moment of inertia. 
These new, heavier rollers could store much more rotational kinetic energy ($E_k = \frac{1}{2}I\omega^2$) at the same angular velocity. 
As a result, when the basketball made contact, the rollers' speed barely decreased, 
allowing them to transfer energy to the ball both efficiently and consistently.
This modification transformed our machine's performance, enabling powerful and repeatable shots that met the competition's requirements. 
Our experience clearly illustrates that the moment of inertia is not merely an abstract textbook concept, 
but a critical, practical tool for solving tangible engineering challenges where controlling and transferring energy is paramount to success.

\vspace{1cm}

% --- 出典 ---
\noindent\large{\textbf{Source}}
\hrule
\begin{itemize}
    \item "Moment of Inertia (慣性モーメント)." \textit{Kanazawa Institute of Technology}. Accessed: Oct. 15, 2025. \url{https://w3e.kanazawa-it.ac.jp/math/physics/category/mechanics/rigidbody_mechanics/rotational_motion/moment_of_inertia.html}
\end{itemize}

\end{document}