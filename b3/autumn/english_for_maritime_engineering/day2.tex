%==============================================================================
% LaTeX Magic Comments
%==============================================================================
% !TEX encoding = UTF-8
% !TEX ts-program = platex

%==============================================================================
% Document Class (ユーザーの環境に合わせてjsarticleを使用)
%==============================================================================
\documentclass[11pt, dvipdfmx]{jsarticle}

%==============================================================================
% Packages (ユーザーの元の環境設定をベースにする)
%==============================================================================

%----- ページ設定 -----
\usepackage[
    top=20mm,
    bottom=20mm,
    left=20mm,
    right=20mm
]{geometry}

%----- フォント設定 -----
\usepackage[T1]{fontenc}
\usepackage{newtxtext}
\usepackage{newtxmath}
\usepackage{newtxtt} % URL表示のエラー対策

%----- 数式関連 -----
\usepackage{amsmath}

%----- 図表・画像関連 -----
\usepackage{graphicx}
\usepackage{here}
\usepackage{float}

%----- その他 -----
\usepackage{setspace} % <<<【追加】行間調整のために追加
\usepackage[hidelinks, unicode]{hyperref} % hidelinksでURLの枠線を消す
\usepackage{pxjahyper}
\usepackage{url}

%==============================================================================
% Main Document Start
%==============================================================================
\begin{document}

% --- ヘッダー・フッターを消去 ---
\pagestyle{empty}

% --- フォントサイズを10.5ptに、行間を少し広めに設定 ---
\fontsize{10.5pt}{12.5pt}\selectfont
\linespread{1.1}

% --- タイトルと氏名、学籍番号 ---
\begin{center}
    \Large{\textbf{Terminology Explanation}}
\end{center}

\vspace{1cm} % 1cmの垂直方向の空白

\noindent Name: Koichiro Koga \\
Student ID: XXXXXXXX
\vspace{1.5cm} % 1.5cmの垂直方向の空白

% --- 用語 ---
\noindent\large{\textbf{Terminology (Technical word)}}
\hrule
\vspace{0.3cm}
\noindent Moment of Inertia

\vspace{1cm} % 1cmの空白

% --- 説明 ---
\noindent\large{\textbf{Explanation}}
\hrule
\vspace{0.3cm}
An aluminum alloy is a metallic material created by combining pure aluminum with other elements, such as copper, magnesium, and silicon. This alloying process enhances the properties of pure aluminum, which is relatively soft. The resulting material is significantly stronger and more durable, while retaining aluminum's most desirable characteristics like its light weight.

The most prominent feature of aluminum alloys is their excellent strength-to-weight ratio. With a density about one-third that of steel, they are indispensable in industries where weight reduction is a primary goal, such as aerospace and naval architecture. For high-speed craft, this property leads to lighter hulls, which allows for greater speeds, improved fuel efficiency, and enhanced stability by lowering the vessel's center of gravity.

Another key property, especially for marine applications, is superior corrosion resistance. Aluminum naturally forms a protective oxide film on its surface, which shields the underlying metal from corrosive elements. Certain alloys, like the 5000 series, are specifically designed for marine environments and offer outstanding resistance to saltwater.

Despite these advantages, aluminum alloys present challenges such as higher costs and the need for specialized welding techniques. However, its excellent workability makes it a highly valuable material in practice. This adaptability is clearly demonstrated in student robotics, where different alloys are selected for specific purposes. For the main structural frame of a robot, A6063 square pipes are often used due to their excellent extrudability and ease of handling. For creating custom, high-precision parts such as gears or brackets that must withstand significant stress, a high-strength A2017 duralumin plate is a common choice. These parts are typically machined from this plate material using a CNC milling machine to achieve the required accuracy. This combination of a strong material and precise manufacturing allows for the creation of lightweight yet robust robot components. This ability to select and process the optimal material for each component makes aluminum alloys a vital and practical choice in engineering.

\vspace{1cm} % 1cmの空白

% --- 出典 ---
\noindent\large{\textbf{Source}}
\hrule
\vspace{0.3cm}
\begin{itemize}
    \item Japan Aluminium Association website, ``Characteristics of Aluminum,'' accessed October 1, 2025.\\
    \url{https://www.aluminum.or.jp/more_knowledge/}
    \item Shima Industries Co., Ltd. website, ``Aluminum Plate,'' accessed October 1, 2025. \\\url{https://simametals.com/home/ita%20arumihtml.html}
\end{itemize}

\end{document}

