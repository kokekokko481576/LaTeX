%==============================================================================
% LaTeXの処理に関するおまじない (マジックコメント)
%==============================================================================
% !TEX encoding = UTF-8
%==============================================================================
% ドキュメントクラス
%==============================================================================
\documentclass[11pt]{ltjsarticle}

%==============================================================================
%【必須級】便利なパッケージたち
%==============================================================================

%----- ページ設定 -----
\usepackage[
    top=20mm,
    bottom=20mm,
    left=20mm,
    right=20mm
]{geometry}

%----- フォント設定 -----
\usepackage[T1]{fontenc}
\usepackage{newtxtext}
\usepackage{courier} 
\usepackage{newtxmath}
\usepackage{textcomp}
\usepackage{newtxtt}

%----- 数式関連 -----
\usepackage{amsmath}
\usepackage[detect-all]{siunitx}

%----- 図表・画像関連 -----
\usepackage{graphicx}
\usepackage{here}
\usepackage{booktabs}
\usepackage{float}

%----- ソースコード表示 -----
\usepackage{listings}
\lstset{
    basicstyle=\ttfamily\small,
    breaklines=true,
    frame=single,
    commentstyle={\itshape \color[gray]{0.5}},
    keywordstyle={\bfseries \color{blue}},
    stringstyle={\color{red}},
    showstringspaces=false,
    numbers=left,
    numberstyle=\tiny\color[gray]{0.5},
    captionpos=b
}

%----- その他便利機能 -----
\usepackage{hyperref}
\usepackage{cleveref}
\crefname{figure}{図}{図}
\crefname{table}{表}{表}
\crefname{section}{第}{第}
\crefname{equation}{式}{式}
\crefname{listing}{リスト}{リスト}

%==============================================================================
% ドキュメント情報
%==============================================================================
\title{day3レポート(海事業界のロードマップについて)}
\author{地球総合工学科 \quad B3 \quad 08C23031 \quad 古賀 光一朗}
\date{\today}

%==============================================================================
% 本文開始
%==============================================================================
\begin{document}

\maketitle

自分が日本海事業界に参加し、上の何れかの技術ロードマップの検討・達成が求められたとして、

\begin{enumerate}
    \item どのロードマップに最も関心があるか?
    \item いつ、何の実現をターゲットとするか?
    \item 2の実現のために鍵となる技術、研究、開発は何か?
\end{enumerate}

\section{関心のあるロードマップ}
自分は大きく、「自動化」に興味があります。操船はもちろん、ファクトリオートメーションまで含めて自動化が必要と考えます。自分は部活動でロボコンをやっていたので、自動機のロマンには非常に惹かれる部分があります。ロボコン時代には機構を作る担当だったのですが、ロボコンをやっていく中で回路や制御の重要性も感じるようになりました。そこで海事業界に照らし合わせて考えたときに、自動化技術の発展が必要不可欠であると考えました。

\section{ターゲットとする実現}
自動化について社会にどれほどの技術が実現しているのかあまり理解していないので、具体的にいつまでにというのは考えづらいのですが、少子高齢化の進む日本において、海事産業を維持していくには早期の実現が必要であると考えます。

\section{鍵となる技術、研究、開発}

操船の自動化とファクトリオートメーションの自動化については分けて考える必要があると考えます。
操船の自動化では、センサー技術や、法律の整備が鍵となると考えます。制御については既にある技術を集結すれば十分可能であると思うので、やはり実現の鍵となるのは、それを実現する高精度なセンサーと、それを受け入れる法整備であると考えます。
ファクトリオートメーションの自動化については、非常に多岐にわたる技術が必要となると考えます。例えば溶接や曲げ加工をミスなくするロボット技術や、それを群制御するシステム、さらにはそのシステムに指令を送るAIにつながった図面管理システム等の様々な技術が必要となると考えます。これらを一つ一つ開発していくことが鍵となると考えます。

\end{document}
