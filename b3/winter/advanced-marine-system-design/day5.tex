%==============================================================================
% LaTeXの処理に関するおまじない (マジックコメント)
%==============================================================================
% !TEX encoding = UTF-8
%==============================================================================
% ドキュメントクラス
%==============================================================================
\documentclass[11pt]{ltjsarticle}

%==============================================================================
%【必須級】便利なパッケージたち
%==============================================================================

%----- ページ設定 -----
\usepackage[
    top=20mm,
    bottom=20mm,
    left=20mm,
    right=20mm
]{geometry}

%----- フォント設定 -----
\usepackage[T1]{fontenc}
\usepackage{newtxtext}
\usepackage{courier} 
\usepackage{newtxmath}
\usepackage{textcomp}
\usepackage{newtxtt}

%----- 数式関連 -----
\usepackage{amsmath}
\usepackage[detect-all]{siunitx}

%----- 図表・画像関連 -----
\usepackage{graphicx}
\usepackage{here}
\usepackage{booktabs}
\usepackage{float}

%----- ソースコード表示 -----
\usepackage{listings}
\lstset{
    basicstyle=\ttfamily\small,
    breaklines=true,
    frame=single,
    commentstyle={\itshape \color[gray]{0.5}},
    keywordstyle={\bfseries \color{blue}},
    stringstyle={\color{red}},
    showstringspaces=false,
    numbers=left,
    numberstyle=\tiny\color[gray]{0.5},
    captionpos=b
}

%----- その他便利機能 -----
\usepackage{hyperref}
\usepackage{cleveref}
\crefname{figure}{図}{図}
\crefname{table}{表}{表}
\crefname{section}{第}{第}
\crefname{equation}{式}{式}
\crefname{listing}{リスト}{リスト}

%==============================================================================
% ドキュメント情報
%==============================================================================
\title{先進海事システム設計論 day5課題}
\author{地球総合工学科 \quad B3 \quad 08C23031 \quad 古賀 光一朗}
\date{\today}

%==============================================================================
% 本文開始
%==============================================================================
\begin{document}

\maketitle
\textbf{問}:
機械学習モデル(AI)だけでなく、物理シミュレーションが依然として重要である理由と、両者を組み合わせるメリットについて、講義内の「CFD代理モデル」や「モデルベース開発(MBD)」の文脈を用いて説明しなさい。

\vspace{5mm}



 近年、機械学習の発展は目覚ましいが、工学的な設計・開発において物理シミュレーションの重要性が失われたわけではない。その理由は、機械学習モデルと物理シミュレーションが依拠する原理の違いにある。
 機械学習モデルは「データ駆動型」であり、過去のデータから相関関係を導き出して予測を行う。そのため、計算速度は極めて高速である一方、学習データに含まれない未知の領域や物理法則を逸脱した挙動に対しては予測精度が保証されず、その判断根拠もブラックボックス化しやすいという課題がある。
 対して、物理シミュレーションは「演繹的」な手法である。ナビエ・ストークス方程式などの支配方程式に基づいて計算を行うため、未知の条件下であっても物理法則に則った妥当な解を導出できるという高い信頼性を持つ。したがって、真に新しい現象の解明や、学習データの生成源としては、物理シミュレーションが依然として不可欠である。

 しかし、物理シミュレーション、特に流体解析(CFD)は計算コストが非常に高く、計算に多大な時間を要するという欠点がある。そこで、両者の利点を組み合わせた手法として「CFD代理モデル(サロゲートモデル)」が注目されている。これは、高精度なCFDによって得られた物理的に正しい解析結果を教師データとして機械学習モデルに学習させる手法である。これにより、物理的な妥当性を担保しつつ、推論時の計算時間を劇的に短縮することが可能となる。

 このCFD代理モデルの利点は、「モデルベース開発(MBD)」の文脈において極めて有効に機能する。MBDでは、システム全体を仮想空間上で結合し、シミュレーションを通じて設計検証を行うが、システム全体の挙動を検証するためには膨大な回数の試行錯誤が必要となる。従来の重いCFDモデルをそのままシステムに組み込むことは計算時間の観点から現実的ではなかったが、軽量化されたCFD代理モデルを用いることで、物理現象を考慮した高精度なシミュレーションをリアルタイムに近い速度で実行することが可能となる。
 結論として、物理シミュレーションによる「物理的正しさ」と、機械学習による「計算の即時性」を組み合わせることは、MBDにおける全体最適化や開発期間の短縮を実現する上で不可欠なアプローチであるといえる。



\end{document}
