%==============================================================================
% LaTeXの処理に関するおまじない
%==============================================================================
% !TEX encoding = UTF-8

%==============================================================================
% ドキュメントクラス
%==============================================================================
\documentclass[11pt]{ltjsarticle}

%==============================================================================
% パッケージ設定
%==============================================================================
\usepackage[top=20mm, bottom=20mm, left=20mm, right=20mm]{geometry}
\usepackage[T1]{fontenc}
\usepackage{newtxtext}
\usepackage{courier}
\usepackage{newtxmath}
\usepackage{textcomp}
\usepackage{newtxtt}
\usepackage{amsmath}
\usepackage[detect-all]{siunitx}
\usepackage{graphicx}
\usepackage{here}
\usepackage{booktabs}
\usepackage{float}
\usepackage{hyperref}
\usepackage{cleveref}

% 図表名の設定
\crefname{figure}{図}{図}
\crefname{table}{表}{表}
\crefname{section}{第}{第}
\crefname{equation}{式}{式}

%==============================================================================
% ドキュメント情報
%==============================================================================
\title{構造解析実習レポート:材料および境界条件の影響比較}
\author{地球総合工学科 \quad 古賀 光一朗}
\date{\today}

%==============================================================================
% 本文開始
%==============================================================================
\begin{document}

\maketitle

%------------------------------------------------------------------------------
% 1. 解析概要
%------------------------------------------------------------------------------
\section{解析対象と条件}

本課題では、梁の材料特性および支持条件(境界条件)の違いが、構造解析結果(変形、応力)に及ぼす影響を検証する。

\subsection{解析モデル}
解析対象の梁(Beam 2)の形状および荷重条件を以下に示す。
\begin{itemize}
    \item 形状: 長さ \SI{1000}{\mm} $\times$ 幅 \SI{20}{\mm} $\times$ 高さ \SI{10}{\mm} の角柱
    \item 荷重: 大きさ \SI{300}{\N} の集中荷重(梁中央に作用)
\end{itemize}

\subsection{要素分割(メッシュ)}
解析精度を確保するため、以下の設定でメッシュ生成を行った。
\begin{itemize}
    \item メッシュサイズ: $2 \times 2 \times 4$ \si{\mm}
    \item 要素数: 約 12,500
    \item 節点数: 約 18,036
\end{itemize}

%------------------------------------------------------------------------------
% 2. ケース1:材料比較
%------------------------------------------------------------------------------
\section{解析ケース1:材料による影響の比較}

同一の形状・境界条件(片側固定)において、材料を「鋼 (Steel)」と「アルミニウム (Aluminum)」に変更した場合の比較を行った。
解析結果を\cref{tab:case1_result}に示す。

\begin{table}[H]
\centering
\caption{材料の違いによる解析結果の比較(片側固定、荷重 \SI{300}{\N})}
\label{tab:case1_result}
\begin{tabular}{lccc}
\toprule
項目 & 鋼 (Steel) & アルミニウム (Alumi) & 比率 (Al/St) \\
\midrule
ヤング率 ($E$) & \SI{210}{\GPa} & \SI{70}{\GPa} & 0.33 \\
最大変位 & \SI{89.07}{\mm} & \SI{267.2}{\mm} & 3.00 \\ % アルミ、鋼は理論比より推測
最大応力 (Mises) & \SI{587.7}{\MPa} & \SI{587.7}{\MPa} & 1.00 \\ % アルミより
最大ひずみ & \num{0.0028} & \num{0.0084} & 3.00 \\ % 値は解析値または計算値(587.7/210000)を入れてください
\bottomrule
\end{tabular}
\end{table}

\begin{figure}[H]
    \centering
    % 左側の画像:片側固定(画像ファイル名を適切なものに変更してください)
    \includegraphics[width=0.45\linewidth]{image (7).png}
    \hfill % 画像の間にスペースを空ける
    % 右側の画像:単純支持(画像ファイル名を適切なものに変更してください)
    \includegraphics[width=0.45\linewidth]{image (6).png}
    
    % キャプション:図の内容に合わせて修正しました
    \caption{鋼における境界条件の違いによる応力分布の比較(左:片側固定、右:単純支持)}
    \label{fig:comparison_boundary}
\end{figure}

\subsection{考察}
解析の結果、ヤング率が約3倍小さいアルミニウムは、鋼に比べて約3倍の変位を示した。
一方で、発生する最大応力(Mises応力)は両者で同等の値となった。
これは、静定構造において応力は外力と形状(断面係数)のみに依存し、材料剛性(ヤング率)には依存しないためである。

%------------------------------------------------------------------------------
% 3. ケース2:境界条件比較
%------------------------------------------------------------------------------
\section{解析ケース2:境界条件による影響の比較}

同一の材料(鋼)を使用し、支持条件を変更して比較を行った。
課題図示の形状に基づき、一端を固定した「片側固定(Cantilever)」と、両端を単純支持した「単純支持(Simply Supported)」の2ケースについて解析した。

\begin{table}[H]
\centering
\caption{境界条件の違いによる解析結果の比較(材料:鋼、荷重 \SI{300}{\N})}
\label{tab:case2_result}
\begin{tabular}{lcc}
\toprule
項目 & 片側固定 & 単純支持 \\
\midrule
拘束条件 & 完全固定(1箇所) & 変位拘束・回転自由(2箇所) \\
最大変位 & \SI{89.07}{\mm} & \SI{17.87}{\mm} \\ %
最大応力 (Mises) & \SI{587.7}{\MPa} & \SI{454.7}{\MPa} \\ %
最大主応力 & - & \SI{226.0}{\MPa} \\ % 理論値比較用
\bottomrule
\end{tabular}
\end{table}

\subsection{結果の理論検証}
単純支持梁(中央集中荷重)として解析した結果、最大変位は \SI{17.87}{\mm} であった。
これに対し、材料力学の公式を用いた理論解を算出する。

\begin{itemize}
    \item 荷重 $P = \SI{300}{\N}$
    \item ヤング率 $E = \SI{210000}{\MPa}$
    \item 断面二次モーメント $I = \frac{bh^3}{12} = \frac{20 \times 10^3}{12} \approx \SI{1666.7}{\mm^4}$
\end{itemize}

最大たわみ $\delta_{max}$ の理論値は以下の通りである。

\begin{equation}
\delta_{max} = \frac{PL^3}{48EI} = \frac{300 \times 1000^3}{48 \times 210000 \times 1666.7} \approx \SI{17.86}{\mm}
\end{equation}

解析値(\SI{17.87}{\mm})は理論値(\SI{17.86}{\mm})と極めて良く一致しており、誤差は0.1\%未満である。
また、最大主応力(\SI{226}{\MPa})についても、理論曲げ応力 $\sigma = \frac{M}{Z} = \frac{75000}{333.3} \approx \SI{225}{\MPa}$ と良い一致を示した。



%------------------------------------------------------------------------------
% 4. 結論
%------------------------------------------------------------------------------
\section{結論}

本解析により以下の知見が得られた。

\begin{enumerate}
    \item 材料変更の影響:剛性(ヤング率)の低下は変位の増大を招くが、静的応力値には影響しない。
    \item 境界条件の影響:片側固定(片持ち)に対し、両端を単純支持することで、最大変位は約1/5(理論比 1/4.8)に低減された。
    \item 妥当性の確認:課題図示の単純支持条件に基づく解析結果は、理論解と高い精度で一致した。
\end{enumerate}

\end{document}
