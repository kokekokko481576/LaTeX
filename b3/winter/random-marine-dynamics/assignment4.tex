%==============================================================================
% LaTeXの処理に関するおまじない (マジックコメント)
%==============================================================================
% !TEX encoding = UTF-8
%==============================================================================
% ドキュメントクラス
%==============================================================================
\documentclass[11pt]{ltjsarticle}

%==============================================================================
%【必須級】便利なパッケージたち
%==============================================================================

%----- ページ設定 -----
\usepackage[
    top=20mm,
    bottom=20mm,
    left=20mm,
    right=20mm
]{geometry}

%----- フォント設定 -----
\usepackage[T1]{fontenc}
\usepackage{newtxtext}
\usepackage{courier}
\usepackage{newtxmath}
\usepackage{textcomp}
\usepackage{newtxtt}

%----- 数式関連 -----
\usepackage{amsmath}
\usepackage[detect-all]{siunitx}

%----- 図表・画像関連 -----
\usepackage{graphicx}
\usepackage{here}
\usepackage{booktabs}
\usepackage{float}

%----- ソースコード表示 -----
\usepackage{listings}
\lstset{
    basicstyle=\ttfamily\small,
    breaklines=true,
    frame=single,
    commentstyle={\itshape \color[gray]{0.5}},
    keywordstyle={\bfseries \color{blue}},
    stringstyle={\color{red}},
    showstringspaces=false,
    numbers=left,
    numberstyle=\tiny\color[gray]{0.5},
    captionpos=b
}

%----- その他便利機能 -----
\usepackage{hyperref}
\usepackage{cleveref}
\crefname{figure}{図}{図}
\crefname{table}{表}{表}
\crefname{section}{第}{第}
\crefname{equation}{式}{式}
\crefname{listing}{リスト}{リスト}

%==============================================================================
% ドキュメント情報
%==============================================================================
\title{ランダム海洋現象学 課題4}
\author{船舶海洋工学コース \quad B3 \quad 08C23031 \quad 古賀 光一朗}
\date{2026年1月15日}

%==============================================================================
% 本文開始
%==============================================================================
\begin{document}

\maketitle

\section{修正PMスペクトラムのゼロアップクロス波周期}

ゼロアップクロス波周期 $T_{02}$ は次式で定義される。

$$
T_{02} = 2\pi \sqrt{\frac{m_0}{m_2}}
$$

修正PMスペクトラム $S_{xx}(\omega) = A\omega^{-5} \exp\{-B\omega^{-4}\}$ に対し、変数変換 $t = B\omega^{-4}$ を用いて各モーメントを計算する。

$$
m_0 = \int_{0}^{\infty} S_{xx}(\omega) d\omega = \frac{A}{4B}
$$

$$
m_2 = \int_{0}^{\infty} \omega^2 S_{xx}(\omega) d\omega = \frac{A\sqrt{\pi}}{4\sqrt{B}}
$$

これらを定義式に代入し整理することで、求める $T_{02}$ が得られる。

$$
T_{02} = 2\pi \sqrt{\frac{A/4B}{A\sqrt{\pi}/4\sqrt{B}}} = 2 \pi^{3/4} B^{-1/4}
$$

\section{ランダム海洋波の横断問題}

\subsection{アップクロスの条件}

変位 $x$ が閾値 $x_c$ を下から上に横断(アップクロス)する条件は以下の通りである。

$$
x = x_c \quad \text{かつ} \quad \dot{x} > 0
$$

\subsection{単位時間あたりの平均横断回数の導出}

同時確率密度関数を $p(x, \dot{x})$ とする。微小時間 $\Delta t$ の間にアップクロスが生じる条件は、変位が $x_c - \dot{x}\Delta t < x < x_c$ (ただし $\dot{x} > 0$)の範囲にあることである。
したがって、横断確率 $P_1$ は次のように近似できる。

$$
P_1 \approx \int_{0}^{\infty} p(x_c, \dot{x}) \dot{x} \Delta t d\dot{x}
$$

単位時間あたりの平均回数 $\nu_+(x_c)$ は、$P_1$ を $\Delta t$ で除して極限をとることで得られる。

$$
\nu_+(x_c) = \lim_{\Delta t \to 0} \frac{P_1}{\Delta t} = \int_{0}^{\infty} p(x_c, \dot{x}) \dot{x} d\dot{x}
$$

\section{狭帯域現象の特徴}

線形で狭帯域な波の特徴として、課題文の「ひとつのゼロアップクロスに対してひとつのピークが現れる」以外に以下が挙げられる。

\begin{itemize}
    \item スペクトラムが特定の周波数帯域に集中し、鋭いピークを持つ。
    \item 瞬時値の確率分布は正規分布、波頂高さ(ピーク値)の確率分布はレーリー分布に従う。
\end{itemize}

\section{波のピーク超過確率の導出}

狭帯域波では、単位時間あたりの全ピーク数 $N_p$ はゼロアップクロス回数 $\nu_+(0)$ にほぼ等しい。また、ピーク値 $x_p$ が $x_c$ を超える頻度は、アップクロス回数 $\nu_+(x_c)$ と等価である。
ガウス過程における $\nu_+(x_c)$ は次式である。

$$
\nu_+(x_c) = \nu_+(0) \exp\left(-\frac{x_c^2}{2m_0}\right)
$$

よって、ピーク値の超過確率 $P(x_p > x_c)$ は以下の通りレーリー分布となる。

$$
P(x_p > x_c) = \frac{\nu_+(x_c)}{N_p} \approx \frac{\nu_+(x_c)}{\nu_+(0)} = \exp\left(-\frac{x_c^2}{2m_0}\right)
$$

\end{document}
