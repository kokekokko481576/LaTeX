%==============================================================================
% LaTeXの処理に関するおまじない (マジックコメント)
%==============================================================================
% !TEX encoding = UTF-8
%==============================================================================
% ドキュメントクラス
%==============================================================================
\documentclass[11pt]{ltjsarticle}

%==============================================================================
%【必須級】便利なパッケージたち
%==============================================================================

%----- ページ設定 -----
\usepackage[
    top=20mm,
    bottom=20mm,
    left=20mm,
    right=20mm
]{geometry}

%----- フォント設定 -----
\usepackage[T1]{fontenc}
\usepackage{newtxtext}
\usepackage{courier} 
\usepackage{newtxmath}
\usepackage{textcomp}
\usepackage{newtxtt}

%----- 数式関連 -----
\usepackage{amsmath}
\usepackage[detect-all]{siunitx}

%----- 図表・画像関連 -----
\usepackage{graphicx}
\usepackage{here}
\usepackage{booktabs}
\usepackage{float}

%----- ソースコード表示 -----
\usepackage{listings}
\lstset{
    basicstyle=\ttfamily\small,
    breaklines=true,
    frame=single,
    commentstyle={\itshape \color[gray]{0.5}},
    keywordstyle={\bfseries \color{blue}},
    stringstyle={\color{red}},
    showstringspaces=false,
    numbers=left,
    numberstyle=\tiny\color[gray]{0.5},
    captionpos=b
}

%----- その他便利機能 -----
\usepackage{hyperref}
\usepackage{cleveref}
\crefname{figure}{図}{図}
\crefname{table}{表}{表}
\crefname{section}{第}{第}
\crefname{equation}{式}{式}
\crefname{listing}{リスト}{リスト}

%==============================================================================
% ドキュメント情報
%==============================================================================
\title{ランダム海洋現象学 課題5}
\author{地球総合工学科 \quad B3 \quad 08C23031 \quad 古賀 光一朗}
\date{2026年1月19日}

%==============================================================================
% 本文開始
%==============================================================================
\begin{document}

\maketitle

\section{有義波高と標準偏差の関係}

線形・狭帯域性を仮定し、波頂高さ $x$ はRayleigh分布に従うとする。
波変位の分散を $m_0$ (標準偏差 $\sigma = \sqrt{m_0}$)とすると、確率密度関数 $p(x)$ は次式となる。

$$
p(x) = \frac{x}{m_0} \exp \left( - \frac{x^2}{2m_0} \right)
$$

上位 $1/3$ の波頂高さの下限値 $x_c$ は、以下の条件より求まる。

$$
\int_{x_c}^{\infty} p(x) dx = \exp \left( - \frac{x_c^2}{2m_0} \right) = \frac{1}{3}
$$

これを解いて、

$$
x_c = \sqrt{2 m_0 \ln 3}
$$

上位 $1/3$ の波頂高さの平均値 $E_{1/3}$ は次式で計算される。

$$
E_{1/3} = 3 \int_{x_c}^{\infty} x p(x) dx = 3 \int_{\sqrt{2 m_0 \ln 3}}^{\infty} \frac{x^2}{m_0} \exp \left( - \frac{x^2}{2m_0} \right) dx
$$

$t = x^2 / (2m_0)$ とおいて積分を行うと、

$$
E_{1/3} \approx 2.002 \sqrt{m_0}
$$

狭帯域性の仮定 $H_{1/3} \approx 2 E_{1/3}$ より、

$$
H_{1/3} \approx 4.004 \sqrt{m_0} \approx 4 \sigma
$$

よって、有義波高は標準偏差の約4倍となる。

\section{修正PMスペクトラムのパラメータ}

修正Pierson-Moskowitzスペクトラムは次式で与えられる。

$$
S_{xx}(\omega) = A \omega^{-5} \exp( -B \omega^{-4} )
$$

0次モーメント $m_0$ および2次モーメント $m_2$ は以下の通り。

$$
m_0 = \int_{0}^{\infty} S_{xx}(\omega) d\omega = \frac{A}{4B}
$$

$$
m_2 = \int_{0}^{\infty} \omega^2 S_{xx}(\omega) d\omega = \frac{\sqrt{\pi} A}{4\sqrt{B}}
$$

有義波高 $H_{1/3}$ とゼロクロスアップ平均波周期 $T_{02}$ は次のように表される。

$$
H_{1/3} \approx 4\sqrt{m_0} = 2\sqrt{\frac{A}{B}}
$$

$$
T_{02} = 2\pi \sqrt{\frac{m_0}{m_2}} = 2\pi (\pi B)^{-1/4}
$$

これらを $A, B$ について解く。まず $T_{02}$ の式より、

$$
B = \frac{16 \pi^3}{T_{02}^4}
$$

これを $H_{1/3}$ の式に代入して $A$ を求める。

$$
A = \frac{B H_{1/3}^2}{4} = \frac{4 \pi^3 H_{1/3}^2}{T_{02}^4}
$$

以上より、パラメータ $A, B$ は以下のようになる。

$$
A = \frac{4 \pi^3 H_{1/3}^2}{T_{02}^4}, \quad B = \frac{16 \pi^3}{T_{02}^4}
$$

\end{document}
