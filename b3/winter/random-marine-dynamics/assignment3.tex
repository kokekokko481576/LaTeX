%==============================================================================
% LaTeXの処理に関するおまじない (マジックコメント)
%==============================================================================
% !TEX encoding = UTF-8
%==============================================================================
% ドキュメントクラス
%==============================================================================
\documentclass[11pt]{ltjsarticle}

%==============================================================================
%【必須級】便利なパッケージたち
%==============================================================================

%----- ページ設定 -----
\usepackage[
    top=20mm,
    bottom=20mm,
    left=20mm,
    right=20mm
]{geometry}

%----- フォント設定 -----
\usepackage[T1]{fontenc}
\usepackage{newtxtext}
\usepackage{courier} 
\usepackage{newtxmath}
\usepackage{textcomp}
\usepackage{newtxtt}

%----- 数式関連 -----
\usepackage{amsmath}
\usepackage[detect-all]{siunitx}

%----- 図表・画像関連 -----
\usepackage{graphicx}
\usepackage{here}
\usepackage{booktabs}
\usepackage{float}

%----- ソースコード表示 -----
\usepackage{listings}
\lstset{
    basicstyle=\ttfamily\small,
    breaklines=true,
    frame=single,
    commentstyle={\itshape \color[gray]{0.5}},
    keywordstyle={\bfseries \color{blue}},
    stringstyle={\color{red}},
    showstringspaces=false,
    numbers=left,
    numberstyle=\tiny\color[gray]{0.5},
    captionpos=b
}

%----- その他便利機能 -----
\usepackage{hyperref}
\usepackage{cleveref}
\crefname{figure}{図}{図}
\crefname{table}{表}{表}
\crefname{section}{第}{第}
\crefname{equation}{式}{式}
\crefname{listing}{リスト}{リスト}

%==============================================================================
% ドキュメント情報
%==============================================================================
\title{ランダム海洋現象学 課題3}
\author{地球総合工学科 \quad B3 \quad 08C23031 \quad 古賀 光一朗}
\date{\today}

%==============================================================================
% 本文開始
%==============================================================================
\begin{document}

\maketitle

\section{数式の導出}

方針:海洋波をフーリエ級数展開し、波スペクトラムを満足するようにフーリエ係数を決定する。

\begin{itemize}
    \item 海洋波のフーリエ級数展開
$$
x(t)=\sum_{n=1}^{\infty}C_{n}\cos(n\Delta\omega t+\epsilon_{n})
$$

    \item $x(t)$ の自己相関関数 $R(\tau)$
$$
R(\tau)=\lim_{T\to\infty}\frac{1}{T}\int_{-T/2}^{T/2}x(t)x(t+\tau)dt=\sum_{n=1}^{\infty}\frac{C_{n}^{2}}{2}\cos(n\Delta\omega\tau)
$$

    \item $x(t)$ のパワースペクトラム $S(\omega)$
$$
S(\omega)=\frac{1}{2\pi}\int_{-\infty}^{\infty}R(\tau)e^{-i\omega\tau}d\tau=\sum_{n=1}^{\infty}\frac{C_{n}^{2}}{4}\{\delta(\omega-n\Delta\omega)+\delta(\omega+n\Delta\omega)\}
$$

    \item パワースペクトラムの微小区間での積分
$$
\int_{n\Delta\omega-\frac{\Delta\omega}{2}}^{n\Delta\omega+\frac{\Delta\omega}{2}}S(\omega)d\omega=\frac{C_{n}^{2}}{4}=S(n\Delta\omega)\Delta\omega
$$

    \item 振幅 $C_{n}$ の決定(Single side spectrumの場合)
$$
C_{n}=\sqrt{2S(n\Delta\omega)\Delta\omega}
$$

    \item 海洋波の時系列モデル
$$
x(t)=\sum_{n=1}^{\infty}\sqrt{2S(n\Delta\omega)\Delta\omega}\cos(n\Delta\omega t+\epsilon_{n})
$$
\end{itemize}

\section{重ね合わせ法で注意すべきこと}

\begin{itemize}
    \item 離散化周波数幅 $\Delta\omega$ に起因する周期性 $T$ が現れる。
$$
T=\frac{2\pi}{\Delta\omega}
$$

    \item 周期性の回避策として $\Delta\omega$ を不等間隔($\Delta\omega_{n}$)にする手法がある。
    \item 不等間隔にした場合でも、各成分の周期 $T_{n}=2\pi/\Delta\omega_{n}$ の最小公倍数で周期性が現れるため、本質的な回避ではない。
    \item 実用上は、周期を極めて長くすることで回避する。
\end{itemize}

\end{document}
