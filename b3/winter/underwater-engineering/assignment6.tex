%==============================================================================
% LaTeXの処理に関するおまじない (マジックコメント)
%==============================================================================
% !TEX encoding = UTF-8
%==============================================================================
% ドキュメントクラス
%==============================================================================
\documentclass[11pt]{ltjsarticle}

%==============================================================================
%【必須級】便利なパッケージたち
%==============================================================================

%----- ページ設定 -----
\usepackage[
    top=20mm,
    bottom=20mm,
    left=20mm,
    right=20mm
]{geometry}

%----- フォント設定 -----
\usepackage[T1]{fontenc}
\usepackage{newtxtext}
\usepackage{courier} 
\usepackage{newtxmath}
\usepackage{textcomp}
\usepackage{newtxtt}

%----- 数式関連 -----
\usepackage{amsmath}
\usepackage[detect-all]{siunitx}

%----- 図表・画像関連 -----
\usepackage{graphicx}
\usepackage{here}
\usepackage{booktabs}
\usepackage{float}

%----- ソースコード表示 -----
\usepackage{listings}
\lstset{
    basicstyle=\ttfamily\small,
    breaklines=true,
    frame=single,
    commentstyle={\itshape \color[gray]{0.5}},
    keywordstyle={\bfseries \color{blue}},
    stringstyle={\color{red}},
    showstringspaces=false,
    numbers=left,
    numberstyle=\tiny\color[gray]{0.5},
    captionpos=b
}

%----- その他便利機能 -----
\usepackage{hyperref}
\usepackage{cleveref}
\crefname{figure}{図}{図}
\crefname{table}{表}{表}
\crefname{section}{問題}{問題}
\crefname{equation}{式}{式}
\crefname{listing}{リスト}{リスト}

%==============================================================================
% ドキュメント情報
%==============================================================================
\title{海中工学 第6回課題}
\author{地球総合工学科 \quad B3 \quad 08C23031 \quad 古賀 光一朗}
\date{\today}

%==============================================================================
% 本文開始
%==============================================================================
\begin{document}

\maketitle

\section{水中グライダーの運動方程式}

鉛直面内の線形運動方程式およびY軸まわりの回転運動の関係式は以下の通りである。

$$
A_{11}\cdot\dot{w}+A_{12}\cdot\dot{q}=B_{11}\cdot w+B_{12}\cdot q-\Delta B
$$

$$
A_{21}\cdot\dot{w}+A_{22}\cdot\dot{q}=B_{21}\cdot w+B_{22}\cdot q+B_{23}\cdot\theta-\Delta B\cdot L_{B}
$$

$$
\dot{\theta} = q
$$

これを状態ベクトル $x=[\begin{matrix}w& q& \theta\end{matrix}]^T$、入力 $u(t)=\Delta B$ とし、
$A\dot{x}=Bx+Cu(t)$ の形式に整理する。
第3式を $0\cdot\dot{w} + 0\cdot\dot{q} + 1\cdot\dot{\theta} = 0\cdot w + 1\cdot q + 0\cdot \theta$ と見なすと、各行列 $A, B, C$ は以下のように求められる。

$$
A = \begin{bmatrix}
A_{11} & A_{12} & 0 \\
A_{21} & A_{22} & 0 \\
0 & 0 & 1
\end{bmatrix}
$$

$$
B = \begin{bmatrix}
B_{11} & B_{12} & 0 \\
B_{21} & B_{22} & B_{23} \\
0 & 1 & 0
\end{bmatrix}
$$

$$
C = \begin{bmatrix}
-1 \\
-L_{B} \\
0
\end{bmatrix}
$$

\section{音源移動時のドップラー効果}

音源が速度 $v_r$ で動き、受波器が固定されている場合(図参照)のドップラー効果の導出を行う。
図中の空欄①~⑦に対応する式は以下の通りである。

\begin{itemize}
    \item[①] 時刻 $t=0$ に発射された第1波が、距離 $D$ を進んで受波器に到達する時刻 $t_1$。
    $$
    t_1 = \frac{D}{c} \quad \cdots \text{①}
    $$

    \item[②] 時刻 $t=T_1$ に第2波を発射する際、音源は速度 $v_r$ で時間 $T_1$ だけ進んでいる。その移動距離は次式となる。
    $$
    v_r T_1 \quad \cdots \text{②}
    $$

    \item[③] このとき、音源と受波器の距離は、初期距離 $D$ から移動距離を引いたものになる。これが第2波の伝搬距離となる。
    $$
    D - v_r T_1 \quad \cdots \text{③}
    $$

    \item[④] 受波器における第1波と第2波の到達時間差、すなわち観測される周期 $\Delta t$ を定義する。
    $$
    \Delta t = t_2 - t_1 \quad \cdots \text{④}
    $$

    \item[⑤] 第2波が受波器に到達する時刻 $t_2$ は、発射時刻 $T_1$ に伝搬時間(距離③ $\div$ 音速 $c$)を加えたものとなる。
    $$
    t_2 = T_1 + \frac{D - v_r T_1}{c} \quad \cdots \text{⑤}
    $$

    \item[⑥] 観測される周期差(時間のずれ)を計算する。
    $$
    \begin{aligned}
    t_2 - t_1 &= \left( T_1 + \frac{D - v_r T_1}{c} \right) - \frac{D}{c} \\
              &= T_1 + \frac{D}{c} - \frac{v_r T_1}{c} - \frac{D}{c} \\
              &= T_1 \left( 1 - \frac{v_r}{c} \right) \quad \cdots \text{⑥}
    \end{aligned}
    $$

    \item[⑦] 観測される周波数 $f_2$ を求める。周波数は周期の逆数 ($f = 1/T$) であるため、
    $$
    f_2 = \frac{1}{t_2 - t_1} = \frac{1}{T_1 (1 - \frac{v_r}{c})} = f_1 \frac{1}{1 - \frac{v_r}{c}}
    $$
    分母分子に $c$ を掛けて整理すると、
    $$
    f_2 = \frac{c}{c - v_r} f_1
    $$
    よって係数は、
    $$
    \frac{c}{c - v_r} \quad \cdots \text{⑦}
    $$
\end{itemize}

\section{音線理論による水平到達距離の導出}

スネルの法則を用い、成層構造を持つ海中での音波伝搬経路を計算する。
各層の音速を $C_1, C_2, C_3$、音線と水平面のなす角(俯角)を $\beta_1, \beta_2, \beta_3$ とする。

\subsection{計算条件}

与えられた条件は以下の通りである。

\begin{itemize}
    \item 音速: $C_1 = \SI{1410}{m/s}, \quad C_2 = \SI{1420}{m/s}, \quad C_3 = \SI{1430}{m/s}$
    \item 層厚: 各層 $\SI{50}{m}$ (水深 $\SI{150}{m}$)
    \item 初期角度: $\beta_1 = \SI{45}{deg}$
    \item 直線伝搬時の水平距離: $x_a = \SI{300}{m}$
\end{itemize}

\subsection{スネルの法則による角度計算}

スネルの法則 $\frac{\cos\beta}{C} = \text{一定}$ より、定数 $K$ を求める。

$$
K = \frac{\cos \SI{45}{deg}}{1410} \approx \num{5.0149e-4}
$$

これより、第2層および第3層での角度 $\beta_2, \beta_3$ を求める。

$$
\beta_2 = \arccos(K \cdot C_2) = \arccos(\num{5.0149e-4} \times 1420) \approx \SI{44.59}{deg}
$$

$$
\beta_3 = \arccos(K \cdot C_3) = \arccos(\num{5.0149e-4} \times 1430) \approx \SI{44.18}{deg}
$$

\subsection{水平到達距離の算出}

各層における水平移動距離 $x_i$ は、層厚 $h=\SI{50}{m}$ を用いて $x_i = h / \tan\beta_i$ で表される。
計算結果を\cref{tab:ray-tracing}に示す。

\begin{table}[H]
    \centering
    \caption{各層における音線計算結果}
    \label{tab:ray-tracing}
    \begin{tabular}{ccccc}
        \toprule
        層 & 音速 $C_i$ (\si{m/s}) & 角度 $\beta_i$ (\si{deg}) & $\tan\beta_i$ & 水平距離 $x_i$ (\si{m}) \\
        \midrule
        1 & 1410 & 45.00 & 1.0000 & 50.00 \\
        2 & 1420 & 44.59 & 0.9858 & 50.72 \\
        3 & 1430 & 44.18 & 0.9718 & 51.45 \\
        \midrule
        合計 (片道) & - & - & - & 152.17 \\
        \bottomrule
    \end{tabular}
\end{table}

海面に戻るまでの全水平距離 $x$ は、片道の合計の2倍となる。

$$
x = 2 \times (50.00 + 50.72 + 51.45) = \SI{304.34}{m}
$$

直線伝搬時の距離 $x_a = \SI{300}{m}$ との差は以下の通りである。

$$
\Delta x = x - x_a = 304.34 - 300 = \SI{4.34}{m}
$$

\end{document}
