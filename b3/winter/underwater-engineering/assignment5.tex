%==============================================================================
% LaTeXの処理に関するおまじない (マジックコメント)
%==============================================================================
% !TEX encoding = UTF-8
%==============================================================================
% ドキュメントクラス
%==============================================================================
\documentclass[11pt]{ltjsarticle}

%==============================================================================
%【必須級】便利なパッケージたち
%==============================================================================

%----- ページ設定 -----
\usepackage[
    top=20mm,
    bottom=20mm,
    left=20mm,
    right=20mm
]{geometry}

%----- フォント設定 -----
\usepackage[T1]{fontenc}
\usepackage{newtxtext}
\usepackage{courier} 
\usepackage{newtxmath}
\usepackage{textcomp}
\usepackage{newtxtt}

%----- 数式関連 -----
\usepackage{amsmath}
\usepackage{bm} % ベクトル表記用
\usepackage[detect-all]{siunitx}

%----- 図表・画像関連 -----
\usepackage{graphicx}
\usepackage{here}
\usepackage{booktabs}
\usepackage{float}

%----- ソースコード表示 -----
\usepackage{listings}
\lstset{
    basicstyle=\ttfamily\small,
    breaklines=true,
    frame=single,
    commentstyle={\itshape \color[gray]{0.5}},
    keywordstyle={\bfseries \color{blue}},
    stringstyle={\color{red}},
    showstringspaces=false,
    numbers=left,
    numberstyle=\tiny\color[gray]{0.5},
    captionpos=b
}

%----- その他便利機能 -----
\usepackage{hyperref}
\usepackage{cleveref}
\crefname{figure}{図}{図}
\crefname{table}{表}{表}
\crefname{section}{第}{第}
\crefname{equation}{式}{式}
\crefname{listing}{リスト}{リスト}

%==============================================================================
% ドキュメント情報
%==============================================================================
\title{海中工学 第5回課題レポート}
\author{地球総合工学科 \quad 船舶海洋工学科目 船舶海洋工学コース \\ B3 \quad 08C23031 \quad 古賀 光一朗}
\date{\today}

%==============================================================================
% 本文開始
%==============================================================================
\begin{document}

\maketitle

\section{課題}
機体固定座標系でのZ方向に関する運動方程式(a)より、線形運動方程式(b)を導出する 
\begin{equation}
 A_{13}=A_{15}=A_{24}=A_{46}=0
\end{equation}

\subsection{微小擾乱の仮定}
定常航走状態からの微小擾乱運動を仮定する。
\begin{align}
 & U_X = U_{X0} + u(t), \quad U_Y = v(t), \quad U_Z = U_{Z0} + w(t) \notag \\
 & \Phi = \varphi(t), \quad \Theta = \theta_0 + \theta(t), \quad \Psi = \psi(t)
\end{align}
角速度の近似:
\begin{equation}
 \omega_X \cong \dot{\varphi} - \dot{\psi}\sin\theta_0, \quad \omega_Y \approx \dot{\theta}, \quad \omega_Z = \dot{\psi}\cos\theta_0
\end{equation}

\section{導出}

運動方程式(a)左辺の各項を線形化する。
\begin{itemize}
 \item $\dot{U}_Z = \dot{w}$
 \item $U_Y\omega_X - U_X\omega_Y \cong v(\dot{\varphi}-\dot{\psi}\sin\theta_0) - (U_{X0}+u)\dot{\theta} \cong -U_{X0}\dot{\theta}$
 \item $-z_G(\omega_X^2+\omega_Y^2) \cong 0$ (2次の微小項)
 \item $-x_G(\dot{\omega}_Y - \omega_X\omega_Z) \cong -x_G(\ddot{\theta} - (\dot{\varphi}\dots)(\dot{\psi}\dots)) \cong -x_G\ddot{\theta}$
 \item $A_{33}\dot{V}_Z \cong A_{33}\dot{w}$
 \item $A_{35}\dot{\omega}_Y \cong A_{35}\ddot{\theta}$
 \item $-(A_{11}V_X + \dots)\omega_Y \cong -(A_{11}U_{X0} + \dots)\dot{\theta} \cong -A_{11}U_{X0}\dot{\theta}$
\end{itemize}
これらを整理すると、左辺は次式となる。
\begin{equation}
 \text{左辺} = (M+A_{33})\dot{w} + (-M x_G + A_{35})\ddot{\theta} - (M+A_{11})U_{X0}\dot{\theta}
\end{equation}

重力・浮力項の線形化:
\begin{equation}
 \cos\Theta \cos\Phi \cong \cos(\theta_0+\theta) \cong \cos\theta_0 - \theta \sin\theta_0
\end{equation}
釣り合い条件 $F_{HZ0} - (\rho g V_B - Mg)\cos\theta_0 = 0$ を用いて定常項を消去すると、右辺は次式となる。
\begin{equation}
 \text{右辺} = F_{HZw} w + F_{HZ\dot{\theta}}\dot{\theta} + (\rho g V_B - Mg)(-\sin\theta_0)\theta + F_{HZ\delta a}\delta a
\end{equation}

左辺と右辺を等置し、移項整理する。
\begin{align}
 (M+A_{33})\dot{w} + (-M x_G + A_{35})\ddot{\theta} &= F_{HZw} w + \{ (M+A_{11})U_{X0} + F_{HZ\dot{\theta}} \}\dot{\theta} \notag \\
 &\quad + (\rho V_B - Mg)g \sin\theta_0 \cdot \theta + F_{HZ\delta a}\delta a
\end{align}
これにより式(b)を得る。

\end{document}
