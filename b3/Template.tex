%==============================================================================
% LaTeX Document Class
%==============================================================================
% jsarticle: 日本語の文書作成で標準的なクラス
% 10.5pt: フォントサイズの指定
\documentclass[10.5pt]{jsarticle}

%==============================================================================
% Page Layout and Fonts
%==============================================================================
% geometry: 用紙サイズや余白を設定するパッケージ
\usepackage[
    top=20mm,
    bottom=20mm,
    left=20mm,
    right=20mm
]{geometry}

% newtxtext, newtxmath: フォントをTimes New Roman系にするためのパッケージ
\usepackage{newtxtext}
\usepackage{newtxmath}

%==============================================================================
% Graphics and Plots
%==============================================================================
% graphicx: 画像を挿入するためのパッケージ
\usepackage{graphicx}
% TikZ/PGFPlots: LaTeX内で高機能なグラフを作成するためのパッケージ
\usepackage{tikz}
\usepackage{pgfplots}
\pgfplotsset{compat=1.18} % PGFPlotsのバージョンを指定

%==============================================================================
% Other useful packages
%==============================================================================
% amsmath: 高度な数式を記述するためのパッケージ
\usepackage{amsmath}
% here: 図表の回り込みを制御し、[H]オプションでその場に強制表示させるパッケージ
\usepackage{here}


%==============================================================================
% Document Information
%==============================================================================
\title{ここにレポートのタイトルを記入}
\author{ここに著者名を記入}
\date{\today} % \today でコンパイルした日の日付が自動で入ります。特定の日付にしたい場合は「2025年10月1日」のように直接記述します。


%==============================================================================
% Main Document
%==============================================================================
\begin{document}

\maketitle % タイトル、著者、日付を出力

\section{このテンプレートの使い方(解説)}

このLaTeXテンプレートは、学術レポートや論文作成の要件を満たすように設計されています。フォントは「Times New Roman」、フォントサイズは「10.5pt」、そして上下左右の余白はすべて「20mm」に統一されています。この設定は、多くの学術機関で推奨される標準的なフォーマットです。

最大の特徴は、外部のソフトウェアを使わずに、LaTeXのコード内で直接グラフを作成できる点と、簡単に画像を挿入できる点です。グラフ作成には強力な「PGFPlots」パッケージを利用しており、関数のプロットから実験データの散布図まで、高品質なグラフを自在に描画できます。例えば、凡例付きの$\sin(x)$と$\cos(x)$のグラフも、後述するサンプルコードのように数行で記述可能です。軸ラベルやタイトル、グリッド線の追加も簡単に行えるため、Excelなどで作成したグラフを貼り付ける手間が省け、文書全体のフォントやスタイルに一貫性を持たせることができます。

画像の挿入には、標準的な「graphicx」パッケージを使用します。これにより、JPEGやPNGなどの一般的な画像ファイルを簡単に文書中に配置できます。キャプション(図の説明)や参照ラベルも簡単に追加できるため、本文中で「図\ref{fig:sample-image}に示すように…」といった形で正確に図を参照することが可能です。画像のサイズ調整も、ページの幅に合わせる「width=\linewidth」や、高さを指定する「height=5cm」など、直感的に行えます。

このように、本テンプレートは基本的な書式設定に加えて、レポート作成で頻繁に必要となる図表の取り扱いを効率化するための機能が組み込まれています。この後のセクションで具体的なコード例を示しているので、それを参考にしながら自身のレポートを作成してみてください。

\section{画像の挿入例}

画像を挿入するには、figure環境と\includegraphicsコマンドを使用します。
画像ファイル(この例では`sample.png`)は、この`.tex`ファイルと同じフォルダに置いてください。
`width=\linewidth`は、画像の横幅を本文の幅に合わせて調整するオプションです。
`\caption{}`で図の説明を、`\label{}`で本文から参照するためのラベルを付けます。

\begin{figure}[H] % [H]オプションで図をこの場所に強制表示
    \centering
    % `example-image-a`はサンプル画像です。手持ちの画像ファイル名(例: sample.png)に置き換えてください。
    \includegraphics[width=0.7\linewidth]{example-image-a} 
    \caption{サンプル画像のキャプション}
    \label{fig:sample-image}
\end{figure}

図\ref{fig:sample-image}は、`graphicx`パッケージを使って挿入した画像の例です。

\section{グラフの作成例}

PGFPlotsパッケージを使うことで、LaTeX内で直接グラフを作成できます。
Excelなどでグラフを作成して画像として貼り付けるよりも、文書全体でフォントや線の太さが統一され、美しい仕上がりになります。

\begin{figure}[H]
    \centering
    \begin{tikzpicture}
        \begin{aSxis}[
            width=0.8\linewidth, % グラフの幅
            height=6cm,          % グラフの高さ
            title={三角関数のグラフ}, % グラフのタイトル
            xlabel={$x$},        % x軸ラベル
            ylabel={$y$},        % y軸ラベル
            xmin=-pi, xmax=pi,   % x軸の範囲
            ymin=-1.2, ymax=1.2, % y軸の範囲
            grid=major,          % グリッド線を表示
            legend pos=outer north east, % 凡例の位置
        ]
        % sin(x)のグラフを追加
        % domain: xの描画範囲, samples: 描画のスムーズさ
        \addplot[blue, thick, domain=-pi:pi, samples=100] {sin(deg(x))};
        \addlegendentry{$\sin(x)$} % 凡例のエントリー

        % cos(x)のグラフを追加
        \addplot[red, thick, domain=-pi:pi, samples=100] {cos(deg(x))};
        \addlegendentry{$\cos(x)$} % 凡例のエントリー
        \end{axis}
    \end{tikzpicture}
    \caption{PGFPlotsで作成したグラフ}
    \label{fig:sample-graph}
\end{figure}

図\ref{fig:sample-graph}は、サインカーブとコサインカーブを描画した例です。

\end{document}
