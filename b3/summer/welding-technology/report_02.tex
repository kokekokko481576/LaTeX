\documentclass{article}
\usepackage{amsmath}
\usepackage{amssymb}
\usepackage{geometry}
\geometry{a4paper, margin=1in}
\begin{document}


\begin{center}
    {\huge 溶接施工法 課題2 \par}
    \vspace{1.5em}
    % ↓ ここにご自身の学籍番号と氏名を入れてください
    {\large 大阪大学 工学部 地球総合工学科 \\ 船舶海洋工学科目 船舶海洋工学コース \\ 08C23031 古賀 光一朗 \par}
    \vspace{1em}
    {\large 2025年7月23日\par}
\end{center}
\vspace{2em}


\section*{冷却速度 (CR) の導出(薄板の場合)}

\begin{equation}
\label{eq:T}
    T(y,t)|_{y=0} = \frac{1}{\sqrt{4\pi kt}} \frac{Q/h}{c\rho}
\end{equation}
\begin{itemize}
    \item $T$: 温度
    \item $t$: 熱源が通過してからの時間
    \item $k$: 熱拡散率
    \item $Q$: 単位長さあたりの溶接入熱
    \item $h$: 板厚
    \item $c$: 比熱
    \item $\rho$: 密度
\end{itemize}

\vspace{5mm}
辺々偏微分して
\begin{equation}
\label{eq:delta T}
    \frac{\partial T}{\partial t}|_{y=0} = \frac{Q/h}{c\rho\sqrt{4\pi k}} \cdot \left(-\frac{1}{2} t^{-3/2}\right) = -\frac{Q/h}{2c\rho\sqrt{4\pi k}} t^{-3/2}
\end{equation}

\subsection*{時間 $t$ の消去}
時間 $t$ を消去し、温度 $T$ を使った式に変形したい。\\
(\ref{eq:T})式を $t^{-1/2}$ について解く
\begin{equation}
\label{eq:t-12}
    t^{-1/2} = \frac{c\rho h \sqrt{4\pi k}}{Q} T(y,t)|_{y=0}
\end{equation}

(\ref{eq:t-12})式を(\ref{eq:delta T})式に代入する
\begin{align*}
\frac{\partial T}{\partial t}|_{y=0} &= -\frac{Q/h}{2c\rho\sqrt{4\pi k}} (t^{-1/2})^3 \\
&= -\frac{Q/h}{2c\rho\sqrt{4\pi k}} \left( \frac{c\rho h \sqrt{4\pi k}}{Q} T(y,t)|_{y=0} \right)^3 \\
&= -\frac{Q/h}{2c\rho\sqrt{4\pi k}} \cdot \frac{(c\rho h)^3 (4\pi k)\sqrt{4\pi k}}{Q^3} (T(y,t)|_{y=0})^3 \\
&= - \frac{Q}{h} \frac{1}{2c\rho} \cdot (4\pi k) \frac{(c\rho h)^3}{Q^3} (T(y,t)|_{y=0})^3 \\
&= -2\pi k \cdot \frac{Q}{h c\rho} \cdot \frac{(c\rho h)^3}{Q^3} (T(y,t)|_{y=0})^3 \\
&= -2\pi k \frac{(c\rho h)^2}{Q^2} (T(y,t)|_{y=0})^3 \\
&= -2\pi k \left( \frac{c\rho h}{Q} \right)^2 (T(y,t)|_{y=0})^3 \quad \text{}
\end{align*}


ここでの温度 $T$ は初期温度 $\theta_0$ からの温度上昇分と考えるため、$T(y,t)|_{y=0} = \theta - \theta_0$ と置き換えられる

よって

\begin{equation}
    CR = \left| \frac{\partial T}{\partial t}|_{y=0} \right| = 2\pi k \left( \frac{c\rho h}{Q} \right)^2 (\theta - \theta_0)^3
\end{equation}
\end{document}