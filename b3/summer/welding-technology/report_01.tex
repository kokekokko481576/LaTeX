\documentclass[a4j]{jsarticle}

\begin{document}

\begin{center}
    {\huge 溶接施工法 宿題回答 \par}
    \vspace{1.5em}
    % ↓ ここにご自身の学籍番号と氏名を入れてください
    {\large 大阪大学 工学部 地球総合工学科 \\ 船舶海洋工学科目 船舶海洋工学コース \\ 08C23031 古賀 光一朗 \par}
    \vspace{1em}
    {\large 2025年7月10日\par}
\end{center}
\vspace{2em}

\section*{TIG溶接法}

\subsection*{概要}
融点が高く消耗しないタングステンを電極として使用し、ArやHeなどの不活性ガスで溶接部を大気から遮蔽(シールド)するアーク溶接法である。必要に応じて、溶加材であるフィラーワイヤを溶融池に補給しながら接合を行う。

\subsection*{利点}
\begin{itemize}
    \item 溶接金属の清浄度が高く、機械的特性および耐食性に優れた継手が得られる。
    \item スラグの発生がごく僅かであり、光沢のある美しいビード外観となる。
    \item ステンレス鋼、アルミニウム合金、チタン合金など、鋼以外の多様な金属材料の溶接に適用できる。
\end{itemize}

\subsection*{欠点}
\begin{itemize}
    \item 溶接能率が悪い。
    \item シールドに用いる不活性ガスが高価である。
\end{itemize}

\section*{サブマージアーク溶接法}

\subsection*{概要}
溶接線の上にあらかじめ散布した粒状のフラックスの中でアークを発生させる溶接法である。このフラックスの中に、消耗電極である溶接ワイヤを連続的に送り込むことで溶接が進行する。

\subsection*{利点}
\begin{itemize}
    \item 大電流を使用できるため溶接能率が良い。
    \item 電流密度が高く熱損失が少ないため、深い溶け込みが得られる。
\end{itemize}

\subsection*{欠点}
\begin{itemize}
    \item 適用できる溶接姿勢が、下向き突合せ溶接、または下向き・水平すみ肉溶接に限定される。
    \item 装置が大型であり、厚板や長大な溶接線を持つ構造物でないと経済的な採算がとりにくい。
\end{itemize}

\section*{(抵抗)スポット溶接}

\subsection*{概要}
接合したい重ね合わせた母材を銅電極で挟み込み、加圧した状態で大電流を流すことで発生する抵抗熱を利用する。この熱によって母材の接触部を局部的に溶融させ、同時に加圧することで接合する方法である。

\subsection*{利点}
\begin{itemize}
    \item 自動車のボディ生産をはじめとして、薄鋼板の接合に多用されている。
\end{itemize}

\subsection*{欠点}
\begin{itemize}
    \item 接合形態が重ね合わせた部分の点状接合となるため、適用できる継手の形状が限定される。
\end{itemize}

\section*{摩擦攪拌接合(FSW)}

\subsection*{概要}
1991年に英国のTWI(The Welding Institute)によって発明された固相接合法の一種である。回転するツールを母材に強く押し付けることで摩擦熱を発生させ、材料を溶融させることなく塑性流動状態にして攪拌し、圧力によって接合する。

\subsection*{利点}
\begin{itemize}
    \item アルミニウム合金の接合に適しており、日本の鉄道車両や欧州の船舶・海洋分野などで豊富な実績がある。
\end{itemize}

\subsection*{欠点}
\begin{itemize}
    \item 接合できる材料は、ツールよりも大幅に硬度が低く、延性の強い材料(現在は主にアルミニウム、マグネシウム、銅)に限られる。
\end{itemize}

\end{document}