\documentclass[12pt]{jarticle}
\usepackage{amsmath}
\usepackage{graphicx}
\usepackage{float}
\usepackage{geometry}
\geometry{a4paper, margin=25mm}

\begin{document}
\begin{center}
    {\huge 溶接施工法 課題3 \par}
    \vspace{1.5em}
    % ↓ ここにご自身の学籍番号と氏名を入れてください
    {\large 大阪大学 工学部 地球総合工学科 \\ 船舶海洋工学科目 船舶海洋工学コース \\ 08C23031 古賀 光一朗 \par}
    \vspace{1em}
    {\large 2025年7月27日\par}
\end{center}

\section*{図の点線は,周辺自由突合せ継手の,板幅方向(溶接
線に直交する方向)応力成分$\sigma_x$の溶接線上分布である.図では溶接線の始終端で$\sigma_x$が圧縮に,溶接線の中央部分で$\sigma_x$が引張りになっている.この理由を説明せよ.}

※図省略\\
\vspace{5mm}

溶接残留応力 $\sigma_x$ が中央で引張り、端部で圧縮となるのは、熱収縮の拘束と応力の自己平衡によるものである。

\paragraph{中央部(引張)}
溶接線が冷却収縮する際、長手方向(y方向)に強い引張応力 $\sigma_y$ が生じる。この $\sigma_y$ のポアソン効果により、板は幅方向(x方向)にも収縮しようとする。しかし、剛性の高い中央部ではこの収縮が拘束されるため、結果としてx方向に引張応力 $\sigma_x$ が発生する。

\paragraph{始終端部(圧縮)}
部材内の残留応力は、全体で力が釣り合うように分布する(自己平衡)。中央部に存在する引張応力を相殺するため、拘束が弱く変形しやすい自由端(始終端部)には、バランスをとる形で圧縮応力 $\sigma_x$ が生じる。

\end{document}