\RequirePackage{plautopatch}
\documentclass[dvipdfmx,a4paper]{jsarticle}
\usepackage{graphicx}

\usepackage{amsmath}
\usepackage{amssymb}
\usepackage{float}

\title{工程管理論}
\author{古賀 光一朗}
\date{2025年7月}

\begin{document}
\maketitle
提出先:fumiali.tanigawa@keihindock.co.jp


締め切り7/21


\section{計画通りに物事を進める難しさの要因とメカニズムを下記4つの視点それぞれで述べよ。}
\subsection{開発・受注の特性から来るもの}
\subsubsection{技術開発について、試作ができない問題}
技術開発についてはあまり本講義では触れらませんでしたが、自分の得意とする分野なので少々執筆させていただきます。\\
自分のいままでやってきた開発は製作とともにともに進められるタイプ、すなわち試作品10号が最終版になるようなものでした。実際その方が製作したもののフィードバックを活かして改良する、PDCAサイクルが回せるので実際の開発にも適していると考えています。しかし、造船の場合はそうもいきません。試作品を作るということは、同時に没になる製品を作り出す行為でもあり、単価の高い造船業でそれをするとお金がかかってしょうがないです。ここはひとつかなり大きな問題点だと考えています。
実際試作をすることで得られる気づきは多くあります。
\begin{itemize}
    \item 設計ミス
    \begin{itemize}
        \item 例えば可動部分の干渉等
    \end{itemize}
    \item 機能不全
    \begin{itemize}
        \item 排出機構の機能が弱い等
    \end{itemize}
    \item 重量超過
\end{itemize}
等々良い製品を作るために見逃せない要素があります。\\
少しであれば作っている途中でも直せるとは思いますが、限界はあります。\\
作り直しを恐れすぎて、私は大学一年生の時に酷いロボットを作りましたが、その時に実感しました。

\subsubsection{受注について}
受注の流れは以下の形でした。
\begin{enumerate}
    \item 引合受領
    \item 見積設計
    \item 見積提出
    \item 見積内容確認
    \item 追加,変更見積
    \item 契約図書作成
\end{enumerate}
造船業の場合顧客の方が立場が強いというのがあり、客に言われて仕様変更や設計の修正が発生する点や、先に値段が決まってしまう点がかなりおかしな点でした。

\subsection{受注後の設計・調達から来るもの}
\subsubsection{設計について}
造船の設計では顧客の意見を取り入れるため手直しが多いというのは大きな問題だと思います。問題になる点はいくつか考えられます。
\begin{itemize}
    \item 設計スピードが遅くなる
    \item 設計者の仕事が増える
\end{itemize}
また、注文の都度都度設計しなおしているので手間がかかる上に注文の特性上そのCADは次回に使いまわせないというのも問題で、毎回の注文ごとに似たような物を設計しなおすというのは本当に無駄の多い作業に思われます。。\\
また、こういったやりなおしにより設計に時間がかかると図面出し遅れにより発注や加工に遅れが生じ、結果として納期に間に合わないことがあります。(これは客の注文を聞いてやっているんだから造船所悪くないやろと思いますが、、)

\subsubsection{調達について}

船舶は大型故に外注する物も多く、事前に発注しておく必要がありますが、発注時期が早すぎたりすると倉庫の在庫を圧迫してしまうし、遅れてしまったら作業が滞ってしまうという問題がありました。
発注忘れはリストを作ったりして防がれると思われますが、そもそもリストに入れ忘れていたり等、落とし穴は至る所にあります。こういうところをAIがチェックしてくれたらいいのになーと日々思いますが、CADのパーツリストもなかなか使いづらい所があるので難しい所です。

\subsection{生産方式・生産計画の特性から来るもの}

\subsubsection{前工程が詰まるとその後は待ち}
\label{maekotei}
講義で寿司の例がありましたが、前の人が加工している間、次の人は待つしかない。ロット生産だとこうなってしまうので、次に早期に引き渡す必要があるが、クレーンが要る等の理由でまとめて引き渡すようなことをしている。

\subsubsection{雑な計画}
\ref{maekotei}で述べた詰まり等を考慮してマージンを用意した計画を立てているせいで、期限日が遅くなり、結果人間の特性として製造が遅くなるというのは想像に易いですが、もしマージンをなくしたせいで実際に詰まりが起きてしまうのも良くない。とこの問題の解決は非常に難しいように思われます。講義で紹介のあった方法等で工数を見積もるというのは確かに有効だと思いますが、15分刻みレベルの予想ができるのかというと怪しいですよね、、、

\subsection{製造現場の特性から来るもの}
\subsubsection{不随作業が多い}
不随作業はきりがないほどたくさんあります。
\begin{itemize}
    \item $CO_2$ガス配線整備
    \item 切りくず清掃
    \item 床に散乱するもの避ける、片づける
    \item クレーン移動\\
    等々
\end{itemize}
片づけ等というのはほとんどの人はやはり嫌いでしょう。仕方なくするのがほとんどでしょう。
集塵なんかは溶接機メーカーが溶接しながら集塵してくれるものを開発すればいいだろうと思うのですが。。。
不随作業で時間を取られるだけでなく、少々苛つくこともあるでしょう。これにより作業スピードはどんどん損なわれていきます。
実際、幣部で片付いた状態とそうでない状態で加工,組立をした場合片付いた状態の方が明らかに早かったです。
原因を分析してみ事前のると以下のようなことが考えられました。
\begin{itemize}
    \item 床が散らかっているせいで移動スピードが著しく低下する
    \item 部品が所定の位置に無いことがあり、作業開始時に最低限の片づけとして周りのモノを押し込めた混沌箱(カオスボックス)から探し出すような事をしているせいで一挙手一投足に時間がかかっていた
    \item 工具すらも紛失してしまい、1時間に1回くらいは大捜索をしていた
    \item 小組立したものを置く場所が無い
    \item 図面に記録するペンが無いからと記録せずにやって後々不足分が発覚、加工しなおし等があった
    \item 進みが悪いせいで進捗に対して疲労が早く、休憩が増えた
    \begin{itemize}
        \item 休憩が増えることで他の人の作業を邪魔していた
    \end{itemize}
\end{itemize}
このように不随作業は、やらなければ作業効率が落ちるし、やるのには時間がかかるということで非常に厄介な存在だと思いました。


\subsubsection{モノが大きく、自動化しにくい}
部材が組み立てられ始める前までは完全に自動化できると考えていますが、組立が始まると部材の形が異なるものが多く、サイズも大きいので、運搬機構やその固定具までも部材ごとにしっかり作り込まないといけないことになります。これは造船業が自動化する際の大きな障壁となってくるはずです。

\section{前項の各要因・メカニズムを解消する方策として考えられることを述べよ。自らの仮説でよい。個々の要因に対する方策、複数要因を包括的に解消する方策、いずれも可。}

\subsection{試作ができない問題について}
製品の品質向上において試作は必須だと考えていますが、この問題については講義最終回でも言っていた事前の作りこみにより解決できると考えています。受注前であれば時間制限はないので時間をかけて開発できます。しかし、実物を作るとお金がかかり過ぎるという問題がありました。もちろん全体アセンブリの開発なら全体を組み立てる必要があるのでその場合に困ります。やはり実物の全てを作って試すのが理想だとは思います。\\
そこですぐに思いつくのは\textbf{小型模型船による実験}と\textbf{仮想シュミレーション}だと思いますが、\\
前者は現在も行われていて、これで推進性能はわかりますが、内部の機能まではなかなか再現できないでしょう。推進性能は最も大事と言ってもいい項目なので最低限というところでしょう。\\
後者はVR技術の進化等により近年期待されている分野だと思われますが、まだまだ実用化にはかなり時間がかかると私は考えています。というのも、計算のコストが高すぎる点に問題があります。まだまだ市販のPCでは6自由度ロボットのシュミレーションでも処理が重たいという状況です。これでは一般のエンジニアがソフトウェア開発できず、それを使う側である我々の仕事も捗りません。軽量で精密な計算のできる物理エンジンと半導体技術の進化に期待するほかありません。まぁどこまで正確にシュミレーションしたいかに依りますが、剛体でほぼ問題なく演算できれば及第点かなと考えています。\\

これらのことを考えると造船業での試作は諦めたほうがよさそうです。\\
そこで次に私の考える理想的なシナリオを書きます。
\subsection{理想的シナリオ}
講義で紹介のあった、先回り設計をして顧客のニーズに合わせて組み替える方法。これをベースに考えます。
\begin{enumerate}
    \item 受注までの準備
    \begin{itemize}
        \item 商船会社の人材を雇用し、先回り設計
        \begin{itemize}
            \item 商船会社の人材のノウハウで需要を先回り
            \item できるだけ共通部品を使い、在庫を抑えるように注意
            \item サブアセンブリの細分化により、修正しやすい設計
        \end{itemize}
        \item 加工技術の研究
        \begin{itemize}
            \item 巨大CNC汎用機(レーザー,プラズマ,フライス)(真空吸引固定型)の導入
            \begin{itemize}
                \item 罫書きから切断、面取りまでを全て自動化
                \item 巨大じゃなくてもベルトコンベア型でも良い
            \end{itemize}
            \item シールド用$CO_2$ガスパイプ地中埋蔵
            \begin{itemize}
                \item どこでも近くのパイプカバー開けて管を挿せば使えるように
            \end{itemize}
        \end{itemize}
        \item 先回り設計の推進性能検証
    \end{itemize}
    \item 受注時の相談
    \begin{itemize}
        \item 設計の評価
        \begin{itemize}
            \item 用意した設計からいい条件がなければダメ出しをしてもらい、再設計
            \item 根幹の設計パターンも変える事も考え、自主的な設計を死守する
            \item 再設計で時間がかかる部分に関しては短期製作で許してもらう
        \end{itemize}
    \end{itemize}
    \item 製作
    \begin{itemize}
        \item 事前の発注
        \begin{itemize}
            \item 顧客がどの設計を選んでも使うことが確定しているものは、もしキャンセルされても次で必ず使うのでさっさと発注してすぐ使えるようにしておく
        \end{itemize}
    \end{itemize}
    \begin{itemize}
        \item タスク細分化
        \begin{itemize}
            \item 細分化によりタスク毎の習得コスト低減
            \item ひとつ前の工程のやり方を習得しておくことで前工程を手伝えるようになる
            \begin{itemize}
                \item 前工程だけなら覚えられる
                \item 2種類の仕事ができるので現場の工員もスキルアップできて楽しい
                \item 前工程を手伝えれば待ち時間をゼロにできる
            \end{itemize}
            \item 細分化されればどこに何分かかるのかわかるようになり、生産管理の密度が上がる
            \begin{itemize}
                \item さらに細かく計画された日程計画
            \end{itemize}
        \end{itemize}
        \item 自動化の導入、レビュー
        \item 短期製作
    \end{itemize}
    \item 引き渡し・メンテナンス
    \begin{itemize}
        \item 顧客から改善点をきく
        \begin{itemize}
            \item 毎回の製作を試作としていくかんじ
            \item 造船所に主権のある設計をしているからできる事
        \end{itemize}
    \end{itemize}
\end{enumerate}
このサイクルを数周すれば良い設計セットができていくはずだと考えています。\\
はじめは設計に時間がかかってしまって他の工程の予定が滞ってしまうので、既存の仕事の少ないベンチャー企業か、潤沢に予算をつぎ込める大企業しかできない事だと思われます。

\end{document}