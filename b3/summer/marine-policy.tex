\RequirePackage{plautopatch}
\documentclass[dvipdfmx,a4paper]{jsarticle}
\usepackage{graphicx}

\usepackage{amsmath}
\usepackage{amssymb}
\usepackage{float}

\title{2025海事政策論講義レポート課題}
\author{古賀 光一朗}
\date{2025年8月6日}

\begin{document}
\maketitle
[提出期限・提出先]
2025年8月6日(水)
CLE

\section{[ 課 題 ]}
海事分野の中で最も関心を有する事項について論じよ。ただし、これまでの講義内容を参考に、当該事項に対する行政の関わり・役割について触れるとともに、関係する当事者の視点を踏まえること。

\subsection{船舶完全自動制御}

あまりこの講義では触れられませんでしたが、わたしは船舶の自動制御に興味があります。それに関して本講義で触れられたのは以下の点でした。
\begin{itemize}
\item 事故時の責任問題
\item 乗組員の教育
\end{itemize}

私がこのテーマに強く惹かれるのは、自動化が船員不足やヒューマンエラーを防ぐ切り札として期待される一方、講義で指摘されたような社会の仕組み自体を変える必要がある点に、大きな面白さと難しさを感じたからです。

自律運航船が事故を起こした場合、今の法律ではAIやシステムの責任を問うことが難しく、国際的なルール作りと国内法の整備が不可欠です。これは行政が主導すべき重要な役割だと考えます。

また、船員さんの仕事も操船からシステム管理へと変わるため、新しい教育や働き方への対応が求められます。船会社にとってはコスト削減の好機ですが、船員さんにとっては雇用への不安と新しいキャリアへの期待が入り混じる大きな変化点です。

結論として、船舶の自動化は技術開発だけでなく、法律や教育といった社会システム全体のアップデートが必要な大きな挑戦だと分かりました。将来、技術者としてこの変革に貢献できるよう、技術と社会の両面から学びを深めていきたいです。


\section{[アンケート]}
\subsection{国家公務員の業務内容の理解は深まりましたか。}
はい、大変深まりました。国の政策として、規模や予算の大きな事業に携われる点に魅力を感じました。特に、国会答弁の準備や、苦労して法案が成立した際には、非常に大きな達成感が得られるのだろうと想像し、素晴らしい仕事だと感じました。

\subsection{興味を引かれた講気はありましたか(複数可)。}
ゼロエミッション船を実現するための補助金制度の調整に関するお話や、IMO(国際海事機関)における国際的なルール作りの舞台裏についてのお話が、普段は知ることのできない内容で非常に興味深かったです。

\subsection{将来の職業選択にあたり、考慮したい事項はありますか。その際、海事局の業務は期待に応えることはできていますか。}
私が職業選択で重視するのは、「楽しい」と感じられるかどうかです。私にとっての「楽しさ」とは、大きな仕事を成し遂げた時の「達成感」や、自分の力が役立っているという「自己効力感」、新しいスキルを身につけられる「成長の機会」、そして給与などによる「生活の安定」によってもたらされるものだと考えています。

今回のお話から、海事局の業務はスケールの大きな仕事を通してこれらの点を多く満たせそうだと感じ、大変魅力的だと思いました。ただ、霞が関勤務となると自宅から遠いかもしれない、といった現実的な働き方の面は少し気になりました。



\end{document}