\documentclass[12pt, a4paper, leqno, dvipdfmx]{jarticle}

% --- パッケージ ---
\usepackage{amsmath, amssymb} % 数式用
\usepackage{graphicx}        % 画像読み込み用
\usepackage{geometry}        % 余白設定
\usepackage{float}           % 図表の位置指定 [H] オプションを使うため
\usepackage{booktabs}        % きれいな表を作るため
\usepackage{multicol}        % 段組用
\usepackage{caption}         % キャプションの調整用
\captionsetup{font=small, labelsep=space} % キャプションのスタイルを少し調整

% --- 用紙設定 ---
\geometry{
    top=25mm,
    bottom=25mm,
    left=25mm,
    right=25mm
}

% --- 文書情報 ---
\title{抵抗推進学 レポート課題}
\author{大阪大学 工学部 地球総合工学科 船舶海洋工学科目 \\ 学籍番号: 08C23031 \\ 氏名: 古賀光一朗}
\date{令和6年7月30日}

% --- 本文開始 ---
\begin{document}

\maketitle

\section*{問1}
\textbf{ケルビン波の作図をしてください.(これはすでに宿題で指示済み.まだの人は至急)}

\vspace{5mm}
本課題については、宿題として作成した資料を別添として提出する。

\newpage

\section*{問2}
\textbf{以下に示す長さ 300mのバルクキャリアの 3m模型を用いて抵抗試験をした。その結果, 形状影響係数 K は, 0.26 と推定された。ただし, 実船の主要目は, 表に示すとおりである。}
\textbf{一方, 造波抵抗は, 以下の式で近似的に表されることがわかった。}
$$
C_w = 
\begin{cases}
    0.13(F_n^3 - 0.001) & (F_n > 0.1) \\
    0 & (F_n \le 0.1)
\end{cases}
$$
\textbf{空気抵抗, 飛沫抵抗, 粗度抵抗などを無視したとき, 船速 12, 13, 14 ノットにおける実船の抵抗(kN)と有効馬力(kW)を求めなさい。ただし, 相当平板の摩擦抵抗係数は, 以下の式を用いるとする。}
$$
C_{F0} = \frac{0.455}{(\log_{10} R_n)^{2.58}}
$$

\vspace{5mm}
\subsection*{与えられた値}
\begin{table}[H]
    \centering
    \caption{実船主要目}
    \begin{tabular}{llr}
        \toprule
        垂線間長 & Lpp (m) & 300 \\
        幅 & B\textsubscript{WL}(m) & 48 \\
        深さ & D(m) & 27 \\
        喫水 & d(m) & 17.7 \\
        排水容積 & $\nabla$(m$^3$) & 219000 \\
        浸水表面積 & S(m$^2$) & 22500 \\
        \bottomrule
    \end{tabular}
\end{table}

\noindent
その他の条件は以下の通りである。
\begin{align*}
    \text{形状影響係数 } k &= 0.26 \\
    \text{重力加速度 } g &= 9.8 \, (\text{m/s}^2) \\
    \text{実船の水の密度 } \rho_S &= 1025 \, (\text{kg/m}^3) 
\end{align*}

\subsection*{計算方法}
講義資料「模型実験と抵抗推定」§25.6 に示される抵抗の成分分離の考え方に基づき、実船の全抵抗係数$C_{TS}$を求める。Froudeの仮説に従い、全抵抗係数$C_{TS}$を粘性抵抗係数$C_V$と造波抵抗係数$C_W$の和で表す。
$$ C_{TS} = C_V(R_n) + C_W(F_n) $$
粘性抵抗係数$C_V$は、形状影響係数$k$と相当平板の摩擦抵抗係数$C_{F0}$を用いて、$C_V=(1+k)C_{F0}$とモデル化する。したがって、
$$ C_{TS} = (1+k)C_{FS} + C_{WS} $$
となる。ここで添え字Sは実船を示し、空気抵抗等は無視するため粗度影響係数$C_A$は0とする。有効馬力$P_E$は$P_E = R_{TS} \cdot V_S$で計算される。

\subsection*{計算過程}
船速の単位を $1 \, \text{knot} = 0.5144 \, \text{m/s}$ として換算し、各船速について計算を行う。

\subsubsection{船速 12 knot の場合}
\begin{itemize}
    \item 船速: $V_S = 12 \times 0.5144 = 6.1728 \, \text{m/s}$
    \item フルード数: $F_{nS} = V_S / \sqrt{g L_{pp}} \approx 0.114$
    \item レイノルズ数: $R_{nS} = V_S L_{pp} / \nu_S \approx 1.85 \times 10^9$
    \item 摩擦抵抗係数: $C_{FS} = 0.455 / (\log_{10}(1.85 \times 10^9))^{2.58} \approx 1.39 \times 10^{-3}$
    \item 造波抵抗係数: $C_{WS} = 0.13(0.114^3 - 0.001) \approx 0.062 \times 10^{-3}$
    \item 全抵抗係数: $C_{TS} = (1+0.26) \times (1.39 \times 10^{-3}) + (0.062 \times 10^{-3}) \approx 1.81 \times 10^{-3}$
    \item 全抵抗: $R_{TS} = \frac{1}{2} \rho_S S V_S^2 C_{TS} \approx 795 \, \text{kN}$
    \item 有効馬力: $P_E = R_{TS} V_S \approx 4905 \, \text{kW}$
\end{itemize}
他の船速についても同様に計算し、結果を以下にまとめる。

\subsection*{結果}
\begin{table}[H]
    \centering
    \caption{計算結果一覧}
    \begin{tabular}{ccc}
        \toprule
        船速 [knot] & 全抵抗 $R_{TS}$ [kN] & 有効馬力 $P_E$ [kW] \\
        \midrule
        12 & 795 & 4905 \\
        13 & 928 & 6204 \\
        14 & 1082 & 7795 \\
        \bottomrule
    \end{tabular}
\end{table}

\newpage
\section*{問3}
\textbf{i) 長さ4[m]の貨物船模型で抵抗/自航試験を行った。種々のデータより, 自航要素 (1-t, 1-w, $\eta_R$) を求めなさい。この模型船の実験に使用したプロペラの単独性能曲線は, 次ページの図のようになることがわかっており, 抵抗/自航試験の結果は以下のようになった。}
\textbf{<曳航時> 曳航速度 V: 1.0[m/s], この時の抵抗 $R_{T0}$: 5.5[N]}
\textbf{<自航時> プロペラ回転数 n=7.8[rps]で1.0[m/s]で自航, この時のスラストは, T=6.3[N], トルクは, Q=0.16[N-m]}
\textbf{ただし, プロペラ直径Dは, 0.145[m], 水の密度$\rho$は, 1000[kg/m$^3$]であるとする。}

\vspace{5mm}
\subsection*{解 i)}
与えられた試験結果とプロペラ単独性能曲線から、自航要素 $1-t, 1-w, \eta_R$ を求める。
\begin{enumerate}
    \item \textbf{推力減少係数 $t$ の計算}\\
    資料「抵抗推進学(プロペラと船体の相互干渉編3)」p.66 (4.3)式に基づき、推力減少係数 $t$ を計算する。
    $$ t = \frac{T - R_{T0}}{T} = \frac{6.3 - 5.5}{6.3} \approx 0.127 \quad \Rightarrow \quad 1-t \approx 0.873 $$
    \item \textbf{伴流係数 $w$ の計算}\\
    資料p.60に示される推力一致法により有効伴流係数$w_e$を求める。まず、自航時の推力係数 $K_{TM}$ を計算する。
    $$ K_{TM} = \frac{T}{\rho n^2 D^4} = \frac{6.3}{1000 \times 7.8^2 \times 0.145^4} \approx 0.233 $$
    この$K_{TM}$をプロペラ単独性能曲線にあてはめ、対応する進捗係数$J$を求める。$K_{T}$の近似式 $0.233 = 0.500 - 0.334J - 0.117J^2$ を解くと、$J \approx 0.658$ を得る。
    これより、プロペラへの有効流入速度 $V_A = J n D \approx 0.744$ m/s。伴流係数 $w$ は、
    $$ w = 1 - \frac{V_A}{V} = 1 - \frac{0.744}{1.0} = 0.256 \quad \Rightarrow \quad 1-w = 0.744 $$
    \item \textbf{相対回転効率 $\eta_R$ の計算}\\
    資料p.60, 70に基づき、相対回転効率 $\eta_R$ を、自航時のトルク係数 $K_{QM}$ と単独時のトルク係数 $K_{Q0}$ の比として計算する。
    $$ K_{QM} = \frac{Q}{\rho n^2 D^5} \approx 0.0401, \quad K_{Q0} \approx 0.0410 \quad (\text{at } J=0.658) $$
    $$ \eta_R = \frac{K_{Q0}}{K_{QM}} = \frac{0.0410}{0.0401} \approx 1.022 $$
\end{enumerate}
\textbf{以上より、自航要素は $1-t \approx 0.873$, $1-w = 0.744$, $\eta_R \approx 1.022$ となる。}

\vspace{1cm}

\textbf{ii) 抵抗と伴流係数には尺度影響がある。実船の抵抗および伴流係数を適当な方法で推定すると 326[kN], 1-w=0.78 であった。この場合, 対応実船速度での長さ 200[m]の実船に必要なエンジンパワーを求めなさい。ここで, 推力減少率と, 推進器効率比には尺度影響がないものとする。さらに伝達効率 $\eta_T$ は 0.98, 海水の密度は 1025[kg/m$^3$]とする。}

\vspace{5mm}
\subsection*{解 ii)}
\begin{enumerate}
    \item \textbf{対応実船速度 $V_S$}: $V_S = V_M \sqrt{L_S/L_M} = 1.0 \sqrt{200/4} \approx 7.07 \, \text{m/s}$。
    \item \textbf{必要推力 $T_S$}: $t$に尺度影響はないため、$t_S = t_M \approx 0.127$。
    $$ T_S = \frac{R_{TS}}{1-t_S} = \frac{326}{1-0.127} \approx 373.4 \, \text{kN} $$
    \item \textbf{プロペラ運転点 ($J_S, n_S$) の特定}: 幾何学的相似を仮定しプロペラ直径をスケールアップ ($D_S = D_M \cdot (L_S/L_M) = 7.25$ m)。プロペラ有効流入速度 $V_{AS} = V_S (1-w_S) \approx 5.515 \, \text{m/s}$。
    $K_T = T_S J_S^2 / (\rho_S V_{AS}^2 D_S^2) \approx 0.228 J_S^2$ と $K_T$特性式を連立して解くと、$J_S \approx 0.813$ を得る。
    これより、$n_S = V_{AS} / (J_S D_S) \approx 0.935$ rps。
    \item \textbf{効率と馬力}: $J_S=0.813$ のとき、$K_T \approx 0.152, K_Q \approx 0.0299$。
    推進効率$\eta_D$と伝達効率$\eta_T$を用いて、エンジンパワー(制動馬力 $P_B$)を求める。
    $\eta_{OS} = J_S K_T / (2\pi K_Q) \approx 0.658$。
    $\eta_H = (1-t_S)/(1-w_S) \approx 1.12$。
    $\eta_{RS} = \eta_{RM} \approx 1.022$。
    $\eta_D = \eta_H \cdot \eta_{OS} \cdot \eta_{RS} \approx 0.753$。
    $$ P_B = \frac{R_{TS} V_S}{\eta_D \eta_T} = \frac{326 \times 10^3 \times 7.07}{0.753 \times 0.98} \approx 3.12 \times 10^6 \, \text{W} $$
\end{enumerate}
\textbf{したがって、必要なエンジンパワーは約 3120 kW となる。}

\newpage
\section*{問4}
\textbf{密度1000[kg/m$^3$]の清水中を直径 5m のプロペラが 5m/s で進んでいる時, プロペラ直径の2倍後方後流のプロペラ固定座標から見た流速は, 7m/s であった。このプロペラが発生しているスラストはいくらか求めなさい。}

\vspace{5mm}
資料「抵抗推進学(推進編テキスト)」p.29に記載の「非回転運動量理論」に基づきスラストを求める。プロペラを通過する流体にのみ着目し、プロペラを「作動円盤」とみなす。
与えられた条件を理論の変数に対応させると、
\begin{itemize}
    \item 無限遠方流速: $U = 5 \, \text{m/s}$
    \item 無限後方流速: $U+u = 7 \, \text{m/s}$
\end{itemize}
となる。したがって、プロペラによって付加された後流速度は $u = (U+u) - U = 7 - 5 = 2\text{m/s}$ である。
運動量理論より、プロペラディスク面での誘導速度 $u'$ は後流速度の半分 $u' = u/2 = 1\text{m/s}$ となる。
スラスト$T$は、単位時間にプロペラを通過する質量と、付加される速度の積で与えられる。
$$ T = (\rho A (U+u')) \cdot u $$
値を代入すると、
$$ T = 1000 \times (\frac{\pi \cdot 5^2}{4}) \times (5+1) \times 2 = 235619 \, \text{N} $$
\textbf{したがって、プロペラが発生するスラストは約 236 kN である。}

\newpage
\section*{問5}
\textbf{以下を説明しなさい。}
\textbf{(ア) 形状影響係数 K (イ) 船殻効率 (ウ) 有効伴流係数と公称伴流係数 (エ) キャビテーションおよびその種類と影響 (オ) 尺度影響 (カ) 船舶ではプロペラを船尾に付けるが, その理由}

\vspace{5mm}
\begin{description}
    \item[(ア) 形状影響係数 K]
    
    船の粘性抵抗を、相当平板の摩擦抵抗から推定する際に用いる補正係数。船のような三次元物体では、純粋な摩擦抵抗に加え、粘性に起因する圧力抵抗(粘性圧力抵抗)が生じる。形状影響係数$k$は、この粘性圧力抵抗を摩擦抵抗に対する比率としてモデル化するものであり、ITTC-1957の性能予測法では粘性抵抗係数$C_V$を$C_V=(1+k)C_F$として扱う。(資料「模型実験と抵抗推定」p.11参照)
    
    \item[(イ) 船殻効率 $\eta_H$] 
    
    プロペラが船体に取り付けられることで生じる流体力学的な干渉効果を総合的に表す効率。船を動かすのに有効な仕事(有効馬力 $P_E=R_T V$)と、プロペラが水に与える仕事(推力馬力 $P_T=T V_A$)の比で定義される。伴流係数$w$と推力減少係数$t$を用いて $\eta_H = (1-t)/(1-w)$ と表される。(資料「抵抗推進学(推進編テキスト)」p.68参照)
    
    \item[(ウ) 有効伴流係数と公称伴流係数] 
    
    どちらも船体後方のプロペラ位置における伴流を表すが、測定方法が異なる。\textbf{公称伴流係数 ($w_n$)}は、プロペラを作動させずに、ピトー管などで船体後方の流速分布を直接計測して得られる伴流の平均値である。一方、\textbf{有効伴流係数 ($w_e$)}は、実際にプロペラを作動させる自航試験において、プロペラ自身を流速計と見なし、その時の推力やトルクから逆算される平均的な伴流である。プロペラの吸引作用により、一般に有効伴流係数の方が公称伴流係数より大きくなる。(資料「抵抗推進学(プロペラと船体の相互干渉編3)」p.59-60参照)
    
    \item[(エ) キャビテーションおよびその種類と影響] 
    
    プロペラ翼面上などの局所的な圧力が、その水の温度における飽和蒸気圧以下になった際に液体が沸騰し、気泡(キャビティ)が発生する現象をいう。\textbf{種類}には、翼面上にシート状に広がるシートキャビテーション、翼端渦の中心に発生する翼端渦キャビテーション、気泡が雲のように集まり激しく崩壊するクラウドキャビテーションなどがある。\textbf{影響}として、
    \\1.気泡崩壊時の衝撃圧によるプロペラ翼面の壊食(エロージョン)、
    \\2.騒音・振動の発生、
    \\3.推力・トルク特性の急激な悪化による性能低下などが挙げられる。(資料「抵抗推進学(推進編テキスト)」p.68, 78参照)
    
    \item[(オ) 尺度影響] 
    
    模型試験の結果から実船の性能を予測する際に生じる、幾何学的相似だけでは説明できない流動現象の差異のこと。船の抵抗において、造波抵抗はフルード数、粘性抵抗はレイノルズ数に支配されるが、模型と実船で両数を同時に一致させることができないために発生する。特に、粘性流体力(摩擦抵抗、伴流など)に顕著に現れ、模型試験から実船性能を推定する際には、経験式や理論式による補正が不可欠となる。(資料「抵抗推進学(プロペラと船体の相互干渉編3)」p.53参照)
    \item[(カ) 船舶でプロペラを船尾に付ける理由] 
    
    主な理由は\textbf{船殻効率$\eta_H$の向上}である。船尾には船体境界層の影響で船速より流速が遅い「伴流」が存在する。プロペラをこの伴流の中で作動させると、流入速度が遅い分、効率的に推力を発生できる(伴流利得、wake gain)。また、舵をプロペラの後流内に配置できるため舵効きが向上するなど、操縦性の観点からも有利である。(資料「抵抗推進学(推進編テキスト)」p.59参照)
\end{description}

\newpage
\section*{問6}
\textbf{近い将来, やってみたい研究について述べよ。}

\vspace{5mm}

自分は部活でロボットコンテストをやっていたこともあり、大きくいうと「海洋空間におけるロボティクス技術の活用」に興味があります。\\具体的にいうと、自動運航船やデジタルシュミレーション、海中ロボット等です。

現状、第5講座の方にお邪魔して「学部学生による自主研究奨励事業」を活用して自動運航の模型を作っています。なので近い将来でいうと卒業研究になりますが、そこではその船を活用した事になるかと考えています。現状案として出ているのはタグボートの自動化という題材ですね。今回修復している船は双頭アジマススラスタを搭載しているのですが、自由度が高く細かい動きに向いている一方、計算等の難易度が高く、学部生のうちに6自由度を使いこなせるようになるのが現状の目標です。

もう少し遠い将来の事を考えると、そこからさらに発展して、群制御で湾内の交通システムを作る事や海底掘削に耐えられる海中ロボットの開発等が興味深いと考えています。

\end{document}