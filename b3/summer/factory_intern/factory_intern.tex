\documentclass[a4j,11pt]{jsarticle}

% --- 基本的なパッケージ ---
\usepackage{amsmath} % 数式用
\usepackage{graphicx} % 図の挿入用
\usepackage{geometry} % 余白設定用
\geometry{top=30mm, bottom=30mm, left=25mm, right=25mm}

% --- タイトル情報 ---
\title{工場実習報告書}

\author{
    大阪大学 工学部 地球総合工学科 \\
    学籍番号:08C23031 \quad 氏名:古賀 光一朗
}

\date{2025年9月23日} % \today の代わりに提出日を直接指定するのも良い

\begin{document}

\maketitle
\tableofcontents
\clearpage

% --- (1) 実習機関・場所 ---
\section{実習機関・場所}
\label{sec:location}
% description環境で見やすく変更
\begin{description}
    \item[実習機関] 古野電気株式会社 SOUTH WING
    \item[場所] 〒662-0843 兵庫県西宮市神祇官町8−1
\end{description}

% --- (2) 実習期間と内容 ---
\section{実習期間と内容}
\label{sec:period_content}
\subsection{実習期間}
2025年9月1日 ~ 2025年9月12日

\subsection{実習内容}
\subsubsection{9月1日:会社紹介とMATLAB入門}
会社説明を受けた後、受け入れ部署へ配属された。業務説明を受け、最初の業務としてMATLABを用いたインテグレータモジュールの製作に取り組んだ。担当したのはEuler法とRunge-Kutta法の実装であった。当初、微分計算まで自動で行う機能を実装しようとしたが、求められていたのは、計算に必要な値を引数として結果を返す単純な機能であった。この手戻りをきっかけに、モジュール設計について自身で調査し、「単一責任の原則」という設計思想の重要性を認識した。

\subsubsection{9月2日:舵モジュール製作}
次に、指令舵角に対し、定格舵角速度の範囲内で動作する舵角計算モジュールを製作した。普段のロボット製作では、モーターへの電圧印加をPWM制御で行うため、物理的な挙動を意識していた。しかし、今回はシミュレータ上での理想的な動作を記述するものであり、その概念の違いに当初は戸惑った。質問を通じて疑問点を解消し、スムーズに作業へ取り掛かることができた。

\subsubsection{9月3日〜9月4日:主機モジュール動特性調査}
指令値に対し、次点での主機回転数を算出する主機モジュールの開発に着手した。まず、主機の動特性をいかなる数式で模擬するかを決定するため、国内外の研究機関における近似手法の調査を行った。9月4日の午後には、本モジュールを使用する開発担当者との会議が設定され、調査結果の報告と今後の方針について議論を交わした。

\subsubsection{9月5日〜9月8日:主機モジュール製作}
調査と会議の結果、主機モジュールは2種類製作する方針となった。一つはエンジンガバナや燃焼時間等を考慮して計算するもの、もう一つは一次遅れ系として簡易的に計算するものである。前者は既存コードの軽量化と可読性向上のためのコメント付与が主な作業であった。特に、計算に不要な変数を含む巨大な構造体を、機能ごとに分割する作業に注力した。後者は単純な構成であったため、速やかに完成させた。

\subsubsection{9月9日〜9月10日:レビュー後の修正作業}
作成したモジュールは、都度、開発担当者によるレビューを受けた。変数名の命名規則やコメントの記述方法、処理の効率化など、多岐にわたる指摘を受け、その都度修正を行った。このレビューと修正の反復作業により、成果物の品質を向上させることができた。

\subsubsection{9月11日〜9月12日:成果発表準備と発表会}
最終日の成果発表会に向け、スライド作成と発表練習を行った。発表会は人事担当者や他のインターンシップ生も参加する形式で実施され、本実習で得た成果について報告した。

% --- (3) 実習で得た成果および知見 ---
\section{実習で得た成果および知見}
\label{sec:results}
今回の実習では、実践的な開発に触れることができ、多くの知見を得た。特に、以下の点は私にとって大きな成果である。

\subsection{MATLABによるデータ解析}
まず、MATLABを習得できたことは大きな収穫である。Euler法やRunge-Kutta法といった計算式を容易に実装できる点や、データを直感的に理解できる形で可視化できる点に、その強力な機能性を感じた。

\subsection{構造体を用いたデータ整理}
主機モジュールの製作において、既存コードで用いられていた巨大な構造体を、機能に必要な変数のみを含む小さな構造体へ分割する作業を行った。この経験を通じ、データを適切に構造化することが、コードの可読性や保守性の向上に直結することを学んだ。

\subsection{モジュール設計の重要性}
インテグレータモジュール開発での手戻りや、巨大な構造体の扱いに試行錯誤した経験から、モジュール設計の重要性を痛感した。「一つのモジュールは一つの役割に専念させる」という単一責任の原則がいかに重要であるかを実感し、初期段階での綿密な設計が、結果として開発全体の効率化に繋がることを学んだ。

\subsection{Mermaidによる思考の可視化}
レビュー提出用のドキュメント作成において、Mermaidというツールを知ることができたのも有益であった。コードの設計意図や処理フローを文章のみで伝達する難しさを感じていたが、Mermaidを用いることで、それらを平易な図として表現できた。思考を可視化することが、情報共有の円滑化に非常に有効な手段であると実感した。

% --- (4) 実習の感想 ---
\section{実習の感想}
\label{sec:thoughts}
今回の実習を通して、実際のシミュレーション開発の現場を体験できたことは、非常に貴重な経験であった。最も大きな発見は、企業における開発が、大学で研究されている学術的な理論や論文に深く根ざしているという点である。主機モジュールの動特性調査では、顧客への説明責任を果たすため、学術的根拠に基づいたアプローチが重視される姿勢を学んだ。

また、コードレビューの重要性も再認識した。自身が記述したコードであっても、第三者による客観的な視点からレビューを受けることで、論理の矛盾や改善点を効率的に発見できた。可読性の高いコードを記述することは、共同開発者への配慮であると同時に、自身のデバッグ作業を助けることにも繋がると実感した。本実習で得た経験を、今後の研究活動や学習に活かしていきたい。
\end{document}