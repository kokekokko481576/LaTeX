%==============================================================================
% LaTeXの処理に関するおまじない (マジックコメント)
%==============================================================================
% !TEX encoding = UTF-8
% !TEX ts-program = platex

%==============================================================================
% ドキュメントクラス
%==============================================================================
\documentclass[11pt, dvipdfmx]{jsarticle}

%==============================================================================
%【必須級】便利なパッケージたち
%==============================================================================

%----- ページ設定 -----
\usepackage[
    top=20mm,
    bottom=20mm,
    left=20mm,
    right=20mm
]{geometry}

%----- フォント設定 -----
\usepackage[T1]{fontenc} %【警告対策①】欧文フォントエンコーディングを指定
\usepackage{newtxtext}
\usepackage{courier} 
\usepackage{newtxmath}
\usepackage{textcomp}
\usepackage{newtxtt}

%----- 数式関連 -----
\usepackage{amsmath}
\usepackage{siunitx}
\usepackage[detect-all]{siunitx} % フォントを自動検出させて警告を消す!

%----- 図表・画像関連 -----
% [demo]オプションで画像ファイルが無くてもコンパイルOKにしてるよ
\usepackage[demo]{graphicx} 
\usepackage{tikz}
\usepackage{pgfplots}
\pgfplotsset{compat=1.18}
\usepackage{here}
\usepackage{booktabs}
\usepackage{float}

%----- ソースコード表示 -----
\usepackage{listings}
\lstset{
    basicstyle=\ttfamily\small,
    breaklines=true,
    frame=single,
    commentstyle={\itshape \color[gray]{0.5}},
    keywordstyle={\bfseries \color{blue}},
    stringstyle={\color{red}},
    showstringspaces=false,
    numbers=left,
    numberstyle=\tiny\color[gray]{0.5},
    captionpos=b
}
% \usepackage{lmodern}

%----- その他便利機能 -----
\usepackage[dvipdfmx, unicode]{hyperref}
\usepackage{pxjahyper} % ブックマークの日本語文字化け対策
\usepackage{cleveref}
\crefname{figure}{図}{図}
\crefname{table}{表}{表}
\crefname{section}{第}{第}
\crefname{equation}{式}{式}

%==============================================================================
% ドキュメント情報
%==============================================================================
\title{ここにレポートのタイトルを記入}
\author{地球総合工学科 \quad B3 \quad XXXXXXXX(学籍番号) \quad 古賀 光一朗}
\date{\today}

%==============================================================================
% 本文開始
%==============================================================================
\begin{document}

\maketitle

\section{画像の挿入例}

画像を挿入するには、\verb|figure|環境と\verb|\includegraphics|コマンドを使います。
\verb|width=\linewidth|は、画像の横幅を本文の幅に合わせるオプションです。
\verb|\cref{fig:sample-image}|のように書くと、自動で「図1」のように表示してくれます。

\begin{figure}[H]
    \centering
    \includegraphics[width=0.7\linewidth]{example-image-a}
    \caption{サンプル画像のキャプション}
    \label{fig:sample-image}
\end{figure}

\cref{fig:sample-image}は、`graphicx`パッケージを使って挿入した画像の例です。

\section{グラフの作成例}

`PGFPlots`を使えば綺麗なグラフが作れます。

\begin{figure}[H]
    \centering
    \begin{tikzpicture}
        \begin{axis}[
            width=0.8\linewidth,
            height=6cm,
            title={三角関数のグラフ},
            xlabel={$x$},
            ylabel={$y$},
            xmin=-pi, xmax=pi,
            ymin=-1.2, ymax=1.2,
            grid=major,
            legend pos=outer north east,
        ]
        \addplot[blue, thick, domain=-pi:pi, samples=100] {sin(deg(x))};
        \addlegendentry{$\sin(x)$}
        
        \addplot[red, thick, domain=-pi:pi, samples=100] {cos(deg(x))};
        \addlegendentry{$\cos(x)$}
        \end{axis}
    \end{tikzpicture}
    \caption{PGFPlotsで作成したグラフ}
    \label{fig:sample-graph}
\end{figure}

\cref{fig:sample-graph}は、サインカーブとコサインカーブを描画した例です。

\section{【追加機能】綺麗な表の作り方}

`booktabs`パッケージを使うと、論文で見るようなプロっぽい表が作れます。
【警告対策】ポイントは、縦の罫線を使わず、横の罫線も\verb|\toprule|, \verb|\midrule|, \verb|\bottomrule|で使い分けること!

\begin{table}[H]
    \centering
    \caption{`booktabs`を使った綺麗な表の例}
    \label{tab:sample-table}
    \begin{tabular}{lcr}
        \toprule
        物質 & 密度 (\si{\kilogram\per\cubic\metre}) & 備考 \\
        \midrule
        水 & \num{1000} & 標準状態 \\
        鉄 & \num{7874} & 常温 \\
        空気 & \num{1.293} & 0℃, 1気圧 \\
        \bottomrule
    \end{tabular}
\end{table}

\cref{tab:sample-table}の密度は`siunitx`パッケージで書いています。

\section{【追加機能】ソースコードの貼り付け}

`listings`パッケージを使えば、PythonやC言語のコードもこの通り!

\begin{lstlisting}[language=Python, caption={Pythonのサンプルコード}, label={lst:py-sample}]
# This is a sample Python code.
def greet(name):
    """This function greets to the person passed in as a parameter"""
    print(f"Hello, {name}!")

if __name__ == '__main__':
    greet('World')
\end{lstlisting}

\cref{lst:py-sample}のように、プログラムの引用も簡単です。

\end{document}